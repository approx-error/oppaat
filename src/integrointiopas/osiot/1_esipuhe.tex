\documentclass[../integrointiopas.tex]{subfiles}
\graphicspath{{\subfix{../kuvat/}}}

\begin{document}
	Jokainen integrointia opiskellut henkilö tietää integroinnin olevan kaikkein yksinkertaisimpia funktioita lukuunottamatta huomattavasti sen käänteisoperaatiota derivointia haastavampaa. Monelle funktiotyypille tai funktioiden yhdistelmälle kun ei ole olemassa yhtä keittokirjamaista reseptiä integroinnin suorittamiseksi. Tämän oppaan tarkoitus on esitellä erilaisia menetelmiä integraalien määrittämiseksi edellämainituissa vaikeammissa tapauksissa ja luoda tätä kautta lukijalle itsevarmuutta laskea eteen tulevia integraaleja sekä toimia kompaktina referenssinä erilaisista integrointimenetelmistä. Oppaassa esitellään integroimismenetelmiä yhden muuttujan funktioille, usean muuttujan funktioille, kompleksiunktioille ja vektorifunktioille sekä muutamia menetelmiä numeerista integrointia varten. Lisäksi oppaassa esitellään muutamia integroinnin sovelluskohteita tilastotieteessä sekä fysiikassa. Oppaan esitietovaatimuksena ovat alkeisfunktiot (polynomi-, juuri-, rationaali-, eksponentti- ja logaritmifunktiot, sekä trigonometriset että hyperboliset funktiot ja niiden käänteisfunktiot), niiden määritelmät ja perusominaisuudet, usean muuttujan funktioiden määritelmät ja perusominaisuudet, raja-arvojen, derivaatan ja integraalin määritelmät sekä yhden, että usean muuttujan tapauksessa, tavallisimmat derivointimenetelmät, kompleksilukujen sekä -funktioiden perusteet ja funktioiden sarjakehitelmät (erityisesti Taylorin sarja). Tätä integrointiopasta ei siis ole tarkoitettu täysin itsenäiseksi lähteeksi integraalilaskentaa opiskelevalle, vaan sen on tarkoitus tutustuttaa lukija erilaisiin integroimismenetelmiin sekä vahvistaa tämän jo olemassaolevaa osaamista integraalilaskennan alueella. Tästä huolimatta kukin integroimismenetelmä on pyritty selittämään mahdollisimman intuitiivisella tavalla ja yleiset tulokset on johdettu intuitiosta eikä toisinpäin. Toivon, että tämä integrointiopas iskostaa lukijaansa motivaation syventää integrointitaitojaan ja toimii ponnahduslautana taitojen syventämiseen. Iloisia lukuhetkiä! \\
	
	\noindent Helsingissä 14.11.2023, \\
	
	\noindent Juuso Kaarela
\end{document}
