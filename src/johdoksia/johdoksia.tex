\documentclass{article}%
\usepackage[T1]{fontenc}%
\usepackage[utf8]{inputenc}%
\usepackage{lmodern}%
\usepackage{textcomp}%
\usepackage{lastpage}%
\usepackage{geometry}%
\geometry{margin=3cm}%
\usepackage[finnish]{babel}%
\usepackage{amsmath}%
\usepackage{amssymb}%
\usepackage{amsthm}%
\usepackage{physics}%
\usepackage{derivative}%
\usepackage{graphicx}%
\usepackage{cancel}%
\usepackage{comment}%
\usepackage{listingsutf8}%
\usepackage{color}%
\usepackage{esint}
\usepackage{bigints}
\usepackage{diagbox}


\numberwithin{equation}{section}
\numberwithin{figure}{section}
\numberwithin{table}{section}	

\newtheorem{theorem}{Lause}[section]
\newtheorem{corollary}{Seuraus}[theorem]
\newtheorem{definition}{Määritelmä}[section]
\newtheorem{remark}{Havainto}[section]


% Unit vector, boldstyle
\newcommand{\unitv}[1]{\mathbf{\hat{#1}}}
% Unit vector, bolstyle (greek)
\newcommand{\unitg}[1]{\boldsymbol{\hat{#1}}}
% Vector, boldstyle
\newcommand{\vtr}[1]{\mathbf{#1}}
% Vector, boldstyle (greek)
\newcommand{\vtg}[1]{\boldsymbol{#1}}

% Vector calculus
% Gradient: use nabla
% Laplacian: use nabla^2
% Divergence:
\newcommand{\divop}{\nabla\cdot}
% Curl:
\newcommand{\curlop}{\nabla\times}

% Rising and falling factorials (Pochhammer symbols)
\newcommand{\risingfact}[1]{^{\overline{#1}}}
\newcommand{\fallingfact}[1]{^{\underline{#1}}}

% Supremum
\newcommand{\Sup}[1]{\underset{#1}{\sup\,}}
% Infimum
\newcommand{\Inf}[1]{\underset{#1}{\inf\,}}

% Inner product
\newcommand{\inprod}[2]{\langle#1,#2\rangle}


% Projektin jako useaan tiedostoon:
\usepackage{subfiles}


\title{Johdoksia fysiikan matematiikasta}%
\author{Juuso Kaarela}%
\date{}
%
\begin{document}%
\normalsize%
\maketitle%

\tableofcontents

\pagebreak

\section{Differentiaalilaskenta}%
\label{sec:diff}%

\subfile{osiot/1_differentiaalilaskenta}

\pagebreak
%
\section{Differentiaaliyhtälötietoa}
\label{sec:difftie}

\subfile{osiot/2_difftietoa}

\pagebreak

%
\section{Tavalliset differentiaaliyhtälöt}%
\label{sec:tdy}%

\subfile{osiot/3_tavalliset_dyt}

\pagebreak

%
\section{Osittaisdifferentiaaliyhtälöt}%
\label{sec:ody}

\subfile{osiot/4_osittais_dyt}

\pagebreak

\section{Vektoriavaruudet ja funktiot vektoreina}
\label{sec:vekt}

\subfile{osiot/5_vektoriavaruudet}

\pagebreak

\section{Funktionaaleista, distribuutioista ja funktionaalianalyysistä}%
\label{sec:funk}

\subfile{osiot/6_funktionaalianalyysi}

\pagebreak

\section{Approksimaatiot, kasvunopeus ja asymptotiikka}
\label{sec:approks}

\subfile{osiot/7_approksimaatiot}

%
\end{document}
