\documentclass{article}
% Encoding and language
\usepackage[T1]{fontenc}
\usepackage[utf8]{inputenc}
\usepackage{lmodern}
\usepackage[british]{babel}
% Page layout
\usepackage{lastpage}
\usepackage{geometry}
\geometry{margin=3cm}
\usepackage{parskip}
% Math
\usepackage{amsmath}
\usepackage{amssymb}
\usepackage{derivative}
\usepackage{cancel}
\usepackage{esint}
% Snippets for other math packages: ppla, pali, pbigi
% Visuals and code
\usepackage{graphicx}
\usepackage{subcaption}
\usepackage{enumerate}
\usepackage{comment}
\usepackage{listingsutf8}
\usepackage{color}
\usepackage{parskip}

\input{~/preferences/tex/math-commands.tex}

\title{\textbf{The Hydrogen Atom \textemdash \ From Basics to Quantum Mechanics}}
\author{Juuso Kaarela}

\begin{document}
    \normalsize
    \maketitle

    \section{Solid Sphere Model}

    Concieved of by John Dalton. Recognized the atom as an indivisible object, and that atoms of a given element are identical. 

    \section{Plum Pudding Model}

    Concieved of by J. J. Thomson. After discovering the electron, Thomson suggested a model of the atom where the negatively charged
    electrons would sit in a positively charged cloud kind of like plums in a pudding.

    \section{Nuclear Model}

    Concieved of by Ernest Rutherford. This was a modification to the plum pudding model made by Rutherford after he discovered
    that positively charges alpha-particles fired at a sheet of gold foil would sometimes pass through with a great deflection angle
    while most of the time would not deflect much. This could only be explained if the atom was mostly empty space with a positive
    nucleus in the middle and the electrons situated far away from the nucleus

    \section{Planetary Model}

    Concieved of by Niels Borh. It was noted that if the electrons orbited around the nucleus, they woul radiate energy according to
    the Larmour formula. This would lead to them collapsing into the nucleus very quickly. Thus atoms could not exist in the way previously
    thought. Bohr fixed this issue by develpoing a model for the atom where the electrons were only allowed to occupy certain discrete
    energy levels. The electrons could them move between the levels by absorbing or emitting energy in discrete packages, which also mathced
    observations. 

    \section{Quantum Model}

    Concieved of by Erwin Schrödinger. As quantum mechanics developed it became clear that the electrons around the nucleus of an atom
    do not behave as point-like particles but more like standing waves whose position at any given moment could not be known. Thus only
    a probability of finding an electron at a given location could be given. These clouds of probability corresponding to the location of
    the electron came to be known as orbitals.

    \section{Deriving the Hydrogen Position Wavefunctions}

    The hydrogen atom is a system of two particles: a proton and an electron. Because there are only two bodies, this system is relatively
    easy to describe mathematically and derivign an analytic solution for the position wavefunction is possible in the first place.
    To begin the derivation, we assume that the hydrogen atom exists in a vacuum with nothing else around to give energy to it or take energy
    away from it. With this assumption in place we can safely say that the hydrogen atom is stable and thus the electron orbiting the proton
    exists in a bound state. We quantify this mathematically by stating that the total energy $E$ of the electron is strictly negative $E < 0$.

    Because the only objects that exist in the system are the proton and the electron, and we assume them to be point-like (meaning no rotational
    inertia), the total energy of the system is the kinetic energies of each of the particles added to the potential enery between them:

    \begin{equation}
        E_{\mathrm{tot}} = K_{\mathrm{p}} + K_{\mathrm{e}} + U_{\mathrm{p}\to\mathrm{e}}
    \end{equation}

    The kinetic energies depend on the respective masses $m_i$ and velocities $\vtr{v}_i = \dot{\vtr{r}}_i$ of each of the particles, while the potential energy is a function
    of the distance between the particles $d = \left| \vtr{r}_{\mathrm{e}} - \vtr{r}_{\mathrm{p}} \right|$:

    \begin{equation}
        \label{eq:tot}
        E_{\mathrm{tot}} = \frac{1}{2}m_{\mathrm{p}}\left| \dot{\vtr{r}}_{\mathrm{p}} \right|^2 + \frac{1}{2}m_{\mathrm{e}}\left| \dot{\vtr{r}}_{\mathrm{e}} \right|^2 + U(\left| \vtr{r}_{\mathrm{e}} - \vtr{r}_{\mathrm{p}} \right|)
    \end{equation}

    Let us define $\vtr{r} = \vtr{r}_{\mathrm{e}} - \vtr{r}_{\mathrm{p}}$. Now, if we set the center of mass of the system to be origin of our reference frame, by the defition of center of mass we have:

    \begin{equation}
        m_{\mathrm{p}}\vtr{r}_{\mathrm{p}} + m_{\mathrm{e}}\vtr{r}_{\mathrm{e}} = \vtr{0}
    \end{equation}

    From this we can derive expressions for $\vtr{r}_{\mathrm{p}}$ and $\vtr{r}_{\mathrm{e}}$ in terms of $\vtr{r}$:

    \begin{equation}
        \vtr{r}_{\mathrm{p}} = \frac{m_{\mathrm{e}}}{m_{\mathrm{p}} + m_{\mathrm{e}}}\vtr{r}, \quad \vtr{r}_{\mathrm{e}} = -\frac{m_{\mathrm{p}}}{m_{\mathrm{p}} + m_{\mathrm{e}}}\vtr{r}
    \end{equation}

    And taking the derivative with respect to time we get:

    \begin{equation}
        \dot{\vtr{r}}_{\mathrm{p}} = \frac{m_{\mathrm{e}}}{m_{\mathrm{p}} + m_{\mathrm{e}}}\dot{\vtr{r}}, \quad \dot{\vtr{r}}_{\mathrm{e}} = -\frac{m_{\mathrm{p}}}{m_{\mathrm{p}} + m_{\mathrm{e}}}\dot{\vtr{r}}
    \end{equation}

    Plugging in these results to equation (\ref{eq:tot}), we get:

    \begin{align}
        E_{\mathrm{tot}} &= \frac{1}{2}m_{\mathrm{p}}\left| \frac{m_{\mathrm{e}}}{m_{\mathrm{p}} + m_{\mathrm{e}}}\dot{\vtr{r}} \right|^2
        + \frac{1}{2}m_{\mathrm{e}}\left| -\frac{m_{\mathrm{p}}}{m_{\mathrm{p}} + m_{\mathrm{e}}}\dot{\vtr{r}} \right|^2 + U(\left| \vtr{r}\right|) \\
        E_{\mathrm{tot}} &= \frac{1}{2}m_{\mathrm{p}}\frac{m_{\mathrm{e}}^2}{(m_{\mathrm{p}} + m_{\mathrm{e}})^2}\left|\dot{\vtr{r}} \right|^2
        + \frac{1}{2}m_{\mathrm{e}}\frac{m_{\mathrm{p}}^2}{(m_{\mathrm{p}} + m_{\mathrm{e}})^2}\left|\dot{\vtr{r}} \right|^2 + U(\left| \vtr{r}\right|) \\
        E_{\mathrm{tot}} &= \frac{1}{2}\frac{m_{\mathrm{p}}m_{\mathrm{e}}^2 + m_{\mathrm{e}}m_{\mathrm{p}}^2}{(m_{\mathrm{p}} + m_{\mathrm{e}})^2}\left|\dot{\vtr{r}} \right|^2
        + U(\left| \vtr{r}\right|) \\
        E_{\mathrm{tot}} &= \frac{1}{2}\frac{m_{\mathrm{p}}m_{\mathrm{e}}(m_{\mathrm{p}} + m_{\mathrm{e}})}{(m_{\mathrm{p}} + m_{\mathrm{e}})^2}\left|\dot{\vtr{r}} \right|^2
        + U(\left| \vtr{r}\right|) \\
        E_{\mathrm{tot}} &= \frac{1}{2}\frac{m_{\mathrm{p}}m_{\mathrm{e}}}{m_{\mathrm{p}} + m_{\mathrm{e}}}\left|\dot{\vtr{r}} \right|^2 + U(\left| \vtr{r}\right|) \\
        \intertext{Getting rid of the norms we get:}
        E_{\mathrm{tot}} &= \frac{1}{2}\frac{m_{\mathrm{p}}m_{\mathrm{e}}}{m_{\mathrm{p}} + m_{\mathrm{e}}}\dot{r}^2 + U(r)
    \end{align}

    Let us denote $\frac{m_{\mathrm{p}}m_{\mathrm{e}}}{m_{\mathrm{p}}}$ with $\mu$. Then we finally get a simple result for the total energy of the system:

    \begin{equation}
        \label{eq:toten}
        \boxed{E_{\mathrm{tot}} = \frac{1}{2}\mu\dot{r}^2 + U(r)}
    \end{equation}

    Next, we need to figure out a description for the potential energy of the system. Using the theory of classical electrostatics we know that by Gauss's law, the total amount of
    electric flux $\Phi_{\mathrm{E}}$ out of a closed surface is proportional to the total electric charge $Q$ contained within the surface:

    \begin{equation}
        \Phi_{\mathrm{E}} \propto Q 
    \end{equation}

    To get this to be an equation, a proportionality constant $\varepsilon_{0}$ is introduced. This physical constant is the permittivity of free space:

    \begin{equation}
        \Phi_{\mathrm{E}} = \frac{Q}{\varepsilon_{0}} 
    \end{equation}

    By definition, the flux of the electric field through some closed surface $S$ is the integral of the component of the electric field intesity $\vtr{E}$
    perpendicular to the surface over the whole surface:

    \begin{align}
        \oiint_{S}\vtr{E}\cdot \odif{\vtr{A}} &= \frac{Q}{\varepsilon_{0}} \\
        \oiint_{S}\vtr{E}\cdot \unitv{n}\odif{A} &= \frac{Q}{\varepsilon_{0}}
    \end{align}

    Then the total charge inside the surface $Q$ is simply the charge of the proton, or one elementary charge $+e$.
    Since the electric field of a proton points radially outward, it can be described in spherical coordinates as $\vtr{E} = E\unitv{r}$:

    \begin{equation}
        \oiint_{S}E\unitv{r}\cdot \unitv{n}\odif{A} = \frac{e}{\varepsilon_{0}}
    \end{equation}

    Let us choose the surface $S$ to be a sphere of radius $r$ centered around the proton. Then the normal vectors $\unitv{n}$ of the surface point
    radially outward and thus correspond to the $\unitv{r}$ vector in spherical coordinates:

    \begin{align}
        \oiint_{S}E\unitv{r}\cdot \unitv{r}\odif{A} &= \frac{e}{\varepsilon_{0}} \\
        \intertext{The dot product vanishes and we get:}
        \oiint_{S}E\odif{A} &= \frac{e}{\varepsilon_{0}} \\
        \intertext{Since we are integrating over a surface of constant radius, the intensity of the electric field is constant over the surface and it can be taken outside of the integral:}
        E\oiint_{S}\odif{A} &= \frac{e}{\varepsilon_{0}} \\
        \intertext{An infinitesimal area element $\odif{A}$ of a spherical shell of radius $r$ is of the form $r^2\sin\theta\odif{\varphi,\theta}$ in spherical coordinates. Thus the integral becomes:}
        E\int_{0}^{\pi}\int_{0}^{2\pi}r^2\sin\theta\odif{\varphi,\theta} &= \frac{e}{\varepsilon_{0}} \\
        \intertext{[INSERT FURTHER EXPLANATIONS]}
        E \cdot 4\pi r^2 &= \frac{e}{\varepsilon_{0}} \\
        E &= \frac{e}{4\pi\varepsilon_{0}r^2} \\
        \vtr{E}(\vtr{r}) &= \frac{e}{4\pi\varepsilon_{0}r^2}\unitv{r}
    \end{align}

    The electric potential energy is defined as the negative work required to bring a test charge from a reference point $r_{\mathrm{ref}}$ to the current point $r$ it is at in the electric field:

    \begin{equation}
        U_{\mathrm{E}} = -W_{r_{\mathrm{ref}}\to r}
    \end{equation}

    [INSERT FURTHER EXPLANATIONS]

    \begin{align}
        U_{\mathrm{E}} &= -\int_{\vtr{r}_{\mathrm{ref}}}^{\vtr{r}} \vtr{F}_{\mathrm{E}}(\vtr{r}')\cdot\odif{\vtr{r}'} \\
        U_{\mathrm{E}} &= -\int_{\vtr{r}_{\mathrm{ref}}}^{\vtr{r}} q\vtr{E}(\vtr{r}')\cdot\odif{\vtr{r}'} \\
        U_{\mathrm{E}} &= -\int_{\vtr{r}_{\mathrm{ref}}}^{\vtr{r}} -e\vtr{E}(\vtr{r}')\cdot\odif{\vtr{r}'} \\
        \intertext{The electric field points along the positive $r$ direction, meaning it can be expressed as: $\vtr{E} = E\vtr{\hat{r}}$. Meanwhile the infinitesimal displacement
        $\odif{\vtr{r}}$ points towards the proton, meaning it is oriented in the $-r$ direction. Thus it can be expressed as $-\odif{r}\vtr{\hat{r}}$}
        U_{\mathrm{E}} &= -\int_{\vtr{r}_{\mathrm{ref}}}^{\vtr{r}} -eE(\vtr{r}')\unitv{r}'\cdot(-\odif{r'}\unitv{r}') \\
        U_{\mathrm{E}} &= -\int_{r_{\mathrm{ref}}}^{r} eE(\vtr{r}')\odif{r'} \\
        U_{\mathrm{E}} &= -\int_{\infty}^{r} eE(\vtr{r}')\odif{r'} \\
        U_{\mathrm{E}} &= -\int_{\infty}^{r} e\frac{e}{4\pi\varepsilon_{0}r'^2}\odif{r'} \\
        U_{\mathrm{E}} &= -\frac{e^2}{4\pi\varepsilon_{0}}\int_{\infty}^{r} \frac{1}{r'^2}\odif{r'} \\
        U_{\mathrm{E}} &= -\frac{e^2}{4\pi\varepsilon_{0}}\left[ -\frac{1}{r'} \right]_{\infty}^{r} \\
        U_{\mathrm{E}} &= -\frac{e^2}{4\pi\varepsilon_{0}}\left[ \lim_{r' \to \infty}\left(-\frac{1}{r'}\right) - \left( -\frac{1}{r} \right) \right] \\
        U_{\mathrm{E}} &= -\frac{e^2}{4\pi\varepsilon_{0}}\left[ 0 + \frac{1}{r} \right] \\
        U_{\mathrm{E}} &= -\frac{e^2}{4\pi\varepsilon_{0}r}
    \end{align}

    Now, by pluggin into equation (\ref{eq:toten}), we get:

    \begin{align}
        E_{\mathrm{tot}} &= \frac{1}{2}\mu\dot{r}^2 - \frac{e^2}{4\pi\varepsilon_{0}r} \\
        \intertext{Let us rewrite the kinectic energy term in terms of the momentum of the system by utilizing $p = \mu \dot{r}$:}
        E_{\mathrm{tot}} &= \frac{1}{2}p\dot{r} - \frac{e^2}{4\pi\varepsilon_{0}r} \\
        \intertext{Multiplying by $1 = \frac{\mu}{\mu}$ we get:}
        E_{\mathrm{tot}} &= \frac{1}{2}p\frac{\mu}{\mu}\dot{r} - \frac{e^2}{4\pi\varepsilon_{0}r}
    \end{align}

    \begin{equation}
        \label{eq:fintot}
        E_{\mathrm{tot}} = \frac{p^2}{2\mu} - \frac{e^2}{4\pi\varepsilon_{0}r}
    \end{equation}

    Equation (\ref{eq:fintot}) corresponds to the hamiltonian of the proton\textendash electron system. In order to describe this situation using quantum mechanics,
    we must perform a quantization. This amounts to replacing the kinetic and potential energies $K$ and $U$ with their corresponding quantum operators $\hat{K}$ and $\hat{U}$ respectively.
    The potential energy operator is simply just the potential energy of the system $\hat{U} = U = U(r) = -\frac{e^2}{4\pi\varepsilon_{0}r}$. The kinetic energy operator on the other hand is
    defined in terms of the momentum operator $\hat{P}$:

    \begin{equation}
        \hat{K} = \frac{\hat{P}^2}{2\mu}
    \end{equation}

    Thus we get, for the quantum mechanical hamiltonian of the system:

    \begin{equation}
        \hat{H} = \frac{\hat{P}^2}{2\mu} - \frac{e^{2}}{4\pi\varepsilon_{0}r}
    \end{equation}

    To be of use, the Hamiltonian should be expressed in terms of the postion $r$. The potential term is already expressed in terms of position which only leaves the kinectic term to be considered.
    In three dimenstions, the momentum operator is defined in terms of the position by the following relation: $\hat{P} = -i\hbar\nabla$, where $\nabla = \left( \pdv{}{x}, \pdv{}{y}, \pdv{}{z} \right)$ is the
    gradient with respect to the spatial coordinates. Thus we get for $\hat{P}^{2}$:

    \begin{align}
        \hat{P}^2 &= \hat{P}\cdot\hat{P} \\
        \hat{P}^2 &= (-i\hbar)(-i\hbar)(\nabla\cdot\nabla) \\
        \hat{P}^2 &= -\hbar^2\nabla^2
    \end{align}

    Thus the final form of the Hamiltonian of the system is:

    \begin{equation}
        \hat{H} = -\frac{\hbar^2}{2\mu}\nabla^2 - \frac{e^{2}}{4\pi\varepsilon_{0}r}
    \end{equation}

    Because the hydrogen atom is a steady state, we can apply the Hamiltonian to the time-independent Schrödinger equation to get the wavefunction of the electron:

    \begin{align}
        \hat{H}\ket{\psi} &= E\ket{\psi} \\
        \left( -\frac{\hbar^2}{2\mu}\nabla^2 - \frac{e^{2}}{4\pi\varepsilon_{0}r} \right)\psi(r, \theta, \varphi) &= E\psi(r, \theta, \varphi)
    \end{align}

    It is to be noted that because of the spherical symmetry of the electric field, we are working in spherical coordinates $r, \theta, \phi$. Expanding the Laplacian $\nabla^2$ in
    spherical coordinates we get:

    \begin{equation}
        \left( -\frac{\hbar^2}{2\mu}\left[ \frac{1}{r^2}\pdv{}{r}\left( r^2\pdv{}{r} \right) + \frac{1}{r^2\sin\theta}\pdv{}{\theta}\left( \sin\theta\pdv{}{\theta} \right) + \frac{1}{r^2\sin^2\theta}\pdv[order=2]{}{\varphi} \right]
        - \frac{e^{2}}{4\pi\varepsilon_{0}r} \right)\psi(r, \theta, \varphi) = E\psi(r, \theta, \varphi)
    \end{equation}

    Distributing the Hamiltonian to the wavefunction we get:

    \begin{equation}
        -\frac{\hbar^2}{2\mu}\left[ \frac{1}{r^2}\pdv{}{r}\left( r^2\pdv{\psi}{r} \right) + \frac{1}{r^2\sin\theta}\pdv{}{\theta}\left( \sin\theta\pdv{\psi}{\theta} \right) + \frac{1}{r^2\sin^2\theta}\pdv[order=2]{\psi}{\varphi} \right]
        - \frac{e^{2}}{4\pi\varepsilon_{0}r}\psi = E\psi
    \end{equation}

    For future convenience, let us multiply the equation by $r^2$ and group terms with only $r$-dependence to the left side of the equation:

    \begin{equation}
        \label{eq:separable}
        -\frac{\hbar^2}{2\mu}\pdv{}{r}\left( r^2\pdv{\psi}{r} \right) - \left( \frac{e^{2}}{4\pi\varepsilon_{0}r} + E \right)r^2\psi = \frac{\hbar^2}{2\mu\sin\theta}\pdv{}{\theta}\left( \sin\theta\pdv{\psi}{\theta} \right)
        + \frac{\hbar^2}{2\mu\sin^2\theta}\pdv[order = 2]{\psi}{\varphi}
    \end{equation}

    Now let us try to solve the differential equation by assuming the solution $\psi = \psi(r, \theta, \varphi)$ is separable, meaning the $r$-, $\theta$-, and $\varphi$-dependencies can be separated into different
    functions with the product of these functions being: $\psi = \psi(r, \theta, \varphi) = R(r)\Theta(\theta)\Phi(\varphi)$. Let us determine $\pdv{\psi}{r}$, $\pdv{\psi}{\theta}$, and $\pdv[order = 2]{\psi}{\varphi}$ in
    terms of this product of functions:

    \begin{itemize}
        \item \textbf{$\pdv{\psi}{r}$:}
            \begin{align}
                \pdv{\psi}{r} &= \pdv{}{r}\left( R(r)\Theta(\theta)\Phi(\varphi) \right) \\
                \pdv{\psi}{r} &= \Theta(\theta)\Phi(\varphi)R'(r)
            \end{align}
        \item \textbf{$\pdv{\psi}{\theta}$:}
            \begin{align}
                \pdv{\psi}{\theta} &= \pdv{}{\theta}\left( R(r)\Theta(\theta)\Phi(\varphi) \right) \\
                \pdv{\psi}{\theta} &= R(r)\Phi(\varphi)\Theta'(\theta)
            \end{align}
        \item \textbf{$\pdv[order=2]{\psi}{\varphi}$:}
            \begin{align}
                \pdv[order=2]{\psi}{\varphi} &= \pdv[order=2]{}{\varphi}\left( R(r)\Theta(\theta)\Phi(\varphi) \right) \\
                \pdv[order=2]{\psi}{\varphi} &= R(r)\Theta(\theta)\Phi''(\varphi)
            \end{align}
    \end{itemize}
      
    Plugging the results into equation (\ref{eq:separable}) we get:
    
    \begin{equation}
        \begin{aligned}
            -\frac{\hbar^2}{2\mu}\pdv{}{r}\left( r^2\Theta(\theta)\Phi(\varphi)R'(r) \right) - \left( \frac{e^{2}}{4\pi\varepsilon_{0}r} + E \right)r^2\psi
            &= \frac{\hbar^2}{2\mu\sin\theta}\pdv{}{\theta}\left( \sin\theta R(r)\Phi(\varphi)\Theta'(\theta) \right) \\
            &\quad + \frac{\hbar^2}{2\mu\sin^2\theta}R(r)\Theta(\theta)\Phi''(\varphi)
        \end{aligned}
    \end{equation}    
    
    Some of the functions are constants w.r.t differentiation meaning they can be taken outside:

    \begin{equation}
        \begin{aligned}
            -\frac{\hbar^2}{2\mu}\Theta(\theta)\Phi(\varphi)\pdv{}{r}\left( r^2R'(r) \right) - \left( \frac{e^{2}}{4\pi\varepsilon_{0}r} + E \right)r^2\psi
            &= \frac{\hbar^2}{2\mu\sin\theta}R(r)\Phi(\varphi)\pdv{}{\theta}\left( \sin\theta \Theta'(\theta) \right) \\
            &\quad + \frac{\hbar^2}{2\mu\sin^2\theta}R(r)\Theta(\theta)\Phi''(\varphi)
        \end{aligned}
    \end{equation}    

    Dividing the equation by $\psi = R\Theta\Phi$ we get:

    \begin{equation}
        -\frac{\hbar^2}{2\mu}\frac{1}{R(r)}\pdv{}{r}\left( r^2R'(r) \right) - \left( \frac{e^{2}}{4\pi\varepsilon_{0}r} + E \right)r^2 = \frac{\hbar^2}{2\mu\sin\theta}\frac{1}{\Theta(\theta)}\pdv{}{\theta}\left( \sin\theta \Theta'(\theta) \right)
        + \frac{\hbar^2}{2\mu\sin^2\theta}\frac{1}{\Phi(\varphi)}\Phi''(\varphi)
    \end{equation}    

    Since the only variable on the left side of the equation is $r$ and it doesn't show up on the right side of the equation, the only way for the sides to be equal is if they are equal to the same constant $A$. Thus
    we get two simultaneous equations:

    \begin{equation}
        \begin{cases}
            -\frac{\hbar^2}{2\mu}\frac{1}{R(r)}\pdv{}{r}\left( r^2R'(r) \right) - \left( \frac{e^{2}}{4\pi\varepsilon_{0}r} + E \right)r^2 &= A \\[0.5em]   
            \frac{\hbar^2}{2\mu\sin\theta}\frac{1}{\Theta(\theta)}\pdv{}{\theta}\left( \sin\theta \Theta'(\theta) \right) + \frac{\hbar^2}{2\mu\sin^2\theta}\frac{1}{\Phi(\varphi)}\Phi''(\varphi) &= A
        \end{cases}
    \end{equation}

    Multiplying the second equation by $\frac{2\mu}{\hbar^2}\sin^2\theta$ we get:

    \begin{equation}
        \frac{\sin\theta}{\Theta(\theta)}\pdv{}{\theta}\left( \sin\theta\Theta'(\theta) \right) + \frac{1}{\Phi(\varphi)}\Phi''(\varphi) = A\frac{2\mu}{\hbar^2}\sin^2\theta
    \end{equation}

    Relabelling $A\frac{2\mu}{\hbar^2} \to A$ and moving terms with only $\varphi$-dependence to the right side and terms with only $\theta$-dependence to the left side we get:

    \begin{equation}
        \frac{\sin\theta}{\Theta(\theta)}\pdv{}{\theta}\left( \sin\theta\Theta'(\theta) \right) + A\sin^2\theta =  -\frac{1}{\Phi(\varphi)}\Phi''(\varphi)
    \end{equation}

    Again, the only variable on the left side is $\theta$ and it doesn't show up on the right side, meaning that the only way for the sides to be equal is if they are equal to the same constant $B$. Thus
    we get more simultaneous equations:

    \begin{equation}
        \begin{cases}
            \frac{\sin\theta}{\Theta(\theta)}\pdv{}{\theta}\left( \sin\theta\Theta'(\theta) \right) + A\sin^2\theta &= B \\[0.5em]    
            -\frac{1}{\Phi(\varphi)}\Phi''(\varphi) &= B
        \end{cases}
    \end{equation}

    Now we have produced three equations, one for each variable:

    \begin{equation}
        \begin{cases}
            -\frac{\hbar^2}{2\mu}\frac{1}{R(r)}\pdv{}{r}\left( r^2R'(r) \right) - \left( \frac{e^{2}}{4\pi\varepsilon_{0}r} + E \right)r^2 &= A \\[0.5em]   
            \frac{\sin\theta}{\Theta(\theta)}\pdv{}{\theta}\left( \sin\theta\Theta'(\theta) \right) + A\sin^2\theta &= B \\[0.5em]    
            -\frac{1}{\Phi(\varphi)}\Phi''(\varphi) &= B
        \end{cases}
    \end{equation}

    Let us make the equations more clear by changing all the partial derivatives to ordinary ones (since the equations only contain one independent variable each) and moving all the terms to the left sides of the
    equations. In addition let us multiply the first equation by $-\frac{2\mu}{\hbar^2R}$, and the third equation by $-1$:

    \begin{equation}
        \begin{cases}
            \odv{}{r}\left( r^2\odv{R}{r} \right) + \frac{2\mu r^2}{\hbar^2}\left( E + \frac{e^2}{4\pi\varepsilon_{0}r} \right)R - AR &= 0 \\[0.5em]
            \frac{\sin\theta}{\Theta}\odv{}{\theta}\left( \sin\theta\odv{\Theta}{\theta} \right) + A\sin^2\theta - B &= 0 \\[0.5em]
            \frac{1}{\Phi}\odv[order = 2]{\Phi}{\varphi} + B &= 0
        \end{cases}
    \end{equation}

    We now have three equations, each describing how a part of the wavefunction $\psi$ behaves. The first equation details the radial behaviour of the function and is known as the radial equation. The
    second equation describes the dependence on the polar angle $\theta$ and is known as the colatitude equation. Finally the third equation depends on the azimuthal angle $\varphi$ and is known as the
    azimuthal equation. Let us solve each of these equations one at a time.

    \subsection{The Azimuthal Equation}

    The azimuthal eqaution is of the form

    \begin{equation}
        \frac{1}{\Phi}\odv[order = 2]{\Phi}{\varphi} + B = 0
    \end{equation}

    By moving the constant $B$ to the r.h.s of the equation and multiplying by $\Phi$, the equation
    takes a very recognizable form:

    \begin{equation}
        \odv[order = 2]{\Phi}{\varphi} = -B\Phi
    \end{equation}

    This is the equation describing a harmonic oscillator and we immediately know that the general solution
    expressed in terms of complex exponentials is of the form:

    \begin{equation}
        \Phi(\varphi) = C_{\varphi}e^{i(\sqrt{B}\varphi + \chi)}
    \end{equation}

    The variable $\chi$ is the phase angle. Since we can always choose the orientation of our coordinate system however we like, we can choose an orientation such that $\chi = 0$
    and we are left with

    \begin{equation}
        \Phi(\varphi) = C_{\varphi}e^{i\sqrt{B}\varphi}
    \end{equation}

    We can reduce this solution further by noting that since the system we are describing is spherically symmetric, the solution must be periodic:
    $\Phi(\varphi + 2n\pi) = \Phi(\varphi)$ for some integer $n$. This means that $\sqrt{B}$ must be an integer. It is commonly denoted $m_{\ell}$ and is known
    as the magnetic quantum number. The index $\ell$ corresponds to another quantum number which $m_{\ell}$ depends on. This connection is explained later
    when studying the colatitude equation. Thus, by plugging in $B = m_{\ell}^2$ we get the solution of the azimuthal equation:

    \begin{equation}
        \label{eq:azimuth}
        \boxed{\Phi(\varphi) = C_{\varphi}e^{im_{\ell}\varphi}}
    \end{equation}

    The constant $C_{\varphi}$ is left as is for know since it will be determined when the wavefunction is later normalized.

    \subsection{The Colatitude Equation}

    The colatitude equation is of the form

    \begin{equation}
        \frac{\sin\theta}{\Theta}\odv{}{\theta}\left( \sin\theta\odv{\Theta}{\theta} \right) + A\sin^2\theta - m_{\ell}^2 = 0
    \end{equation}

    where we have plugged in $B = m_{\ell}^2$ from the solution of the azimuthal equation. We begin the solution by rearranging some terms. Multiplying the
    equation by $\frac{\Theta}{\sin^2\theta}$ we get: 

    \begin{equation}
        \frac{1}{\sin\theta}\odv{}{\theta}\left( \sin\theta\odv{\Theta}{\theta} \right) + \left( A - \frac{m_{\ell}^2}{\sin^2\theta} \right)\Theta = 0
    \end{equation}

    Upon evaluating the derivatives we get:

    \begin{equation}
        \frac{1}{\sin\theta}\left(\sin\theta\odv[order=2]{\Theta}{\theta} + \cos\theta\odv{\Theta}{\theta}\right) + \left( A - \frac{m_{\ell}^2}{\sin^2\theta} \right)\Theta = 0
    \end{equation}

    Now we perform a change of variables where we let $u = \cos\theta \iff \odv{u}{\theta} = -\sin\theta$. The derivatives change according to the chain rule:

    \begin{align}
        \odv{\Theta}{\theta} &= \odv{\Theta}{u}\odv{u}{\theta} \\
                             &= -\sin\theta\odv{\Theta}{u}
    \end{align}

    \begin{align}
        \odv[order = 2]{\Theta}{\theta} &= \odv{}{\theta}\left( \odv{\Theta}{\theta} \right) \\
                                        &= \odv{}{\theta}\left( -\sin\theta\odv{\Theta}{u} \right) \\
                                        &= -\sin\theta\odv{}{\theta}\left( \odv{\Theta}{u} \right) - \cos\theta\odv{\Theta}{u} \\
                                        \intertext{From the chain rule we get:}
                                        &= -\sin\theta \odv[order = 2]{\Theta}{u}\odv{u}{\theta} - \cos\theta\odv{\Theta}{u} \\
                                        &= -\sin\theta \odv[order = 2]{\Theta}{u}(-\sin\theta) - \cos\theta\odv{\Theta}{u} \\
                                        &= \sin^2\theta \odv[order = 2]{\Theta}{u} - \cos\theta\odv{\Theta}{u}
    \end{align}

    Plugging in these results we get:

    \begin{align}
        \frac{1}{\sin\theta}\left(\sin\theta\left[ \sin^2\theta\odv[order = 2]{\Theta}{u} - \cos\theta\odv{\Theta}{u} \right] + 
        \cos\theta\left[ -\sin\theta\odv{\Theta}{u} \right]\right) + \left( A - \frac{m_{\ell}^2}{\sin^2\theta} \right)\Theta &= 0 \\
        \frac{1}{\cancel{\sin\theta}}\left(\sin^{\cancel{3}}\theta\odv[order = 2]{\Theta}{u} - \cancel{\sin\theta}\cos\theta\odv{\Theta}{u} -  
        \cancel{\sin\theta}\cos\theta\odv{\Theta}{u}\right) + \left( A - \frac{m_{\ell}^2}{\sin^2\theta} \right)\Theta &= 0 \\
        \sin^{2}\theta\odv[order = 2]{\Theta}{u} - \cos\theta\odv{\Theta}{u} - \cos\theta\odv{\Theta}{u} + \left( A - \frac{m_{\ell}^2}{\sin^2\theta} \right)\Theta &= 0 \\
        \sin^{2}\theta\odv[order = 2]{\Theta}{u} - 2\cos\theta\odv{\Theta}{u} + \left( A - \frac{m_{\ell}^2}{\sin^2\theta} \right)\Theta &= 0 \\
        \intertext{Expressing the sines in terms of cosines by utilizing $\sin^2\theta + \cos^2\theta = 1 \iff \sin^2\theta = 1 - \cos^2\theta$ we get:}
        (1 - \cos^2\theta)\odv[order = 2]{\Theta}{u} - 2\cos\theta\odv{\Theta}{u} + \left( A - \frac{m_{\ell}^2}{1 - \cos^2\theta} \right)\Theta &= 0 \\
        \intertext{Finally plugging in the change of variables $u = \cos\theta$ we get:}
        (1 - u^2)\odv[order = 2]{\Theta}{u} - 2u\odv{\Theta}{u} + \left( A - \frac{m_{\ell}^2}{1 - u^2} \right)\Theta &= 0
    \end{align}

    The last equation is known as the associated Legendre equation, which is not an easy equation to solve. The method of solution is not very intuitive but rather
    involves a great deal of algebraic manipulations which seem to come out of nowhere when encountered for the first time and while they are understandable once
    one gets to know them, one can't help but shake the feeling that to have come up with any of the algebraic manipulations in the first place would have been a miracle.

    What follows is a solution of the associated Legendre equation with a tour of the relevant mathematical techniques used to solve it

    \subsubsection{Associated Legendre Equation Part 1: The Legendre Polynomials}
    
    The Associated Legendre equation calls first for a discussion of the ordinary Legendre equation, which is of the form:

    \begin{equation}
        \label{eq:legendre}
        (1 - x^2)\odv[order = 2]{f}{x} - 2x\odv{f}{x} + l(l + 1)f = 0
    \end{equation}

    for some real number, but most commonly a natural number $l$. This equation followed from trying to describe the inverse distance between two points, a common concept in physics, using
    spherical coordinates. For example, when calculating the gravitational of electric potential of a mass or charge distribution, one encounters terms of the form

    \begin{equation}
        \frac{1}{\left|\vtr{r} - \vtr{r}'\right|}
    \end{equation}

    where $\vtr{r}$ and $\vtr{r}'$ denote the positions of the mass/charge and an observation point respectively. The vectors $\vtr{r}$, $\vtr{r}'$ and $\vtr{r} - \vtr{r}'$ form a triangle and
    if the angle between $\vtr{r}$ and $\vtr{r}'$ is denoted with $\theta$, we have (by the cosine rule) the following relation:

    \begin{equation}
        \left|\vtr{r} - \vtr{r}'\right|^2 = \left|\vtr{r}\right|^2 + \left|\vtr{r}'\right|^2 - 2\left| \vtr{r} \right|\left| \vtr{r}' \right|\cos\theta = r^2 + r'^2 - 2rr'\cos\theta
    \end{equation}

    Thus the inverse distance $\frac{1}{\left| \vtr{r} - \vtr{r}' \right|}$ becomes:

    \begin{equation}
        \frac{1}{\left| \vtr{r} - \vtr{r}' \right|} = \frac{1}{\sqrt{r^2 + r'^2 - 2rr'\cos\theta}}
    \end{equation}

    Let us choose a coordinate system such that the vector $\vtr{r}'$ points along the z-axis and is normalized, ie $r' = 1$. This lets us perform a substitution $t = \cos\theta$, since the angle measured
    from the z-axis is the polar angle $\theta$. Remembering that since the range of the cosine function is $[-1, 1]$, our substitution requires the value of $t$ to also be confined to the range $[-1, 1]$.
    We thus get a function $g(t, r)$ describing the inverse distance:

    \begin{equation}
        g(t, r) = \frac{1}{\sqrt{1 + r^2 - 2rt}}
    \end{equation}
    
    To get a more usable form of the function, let us expand it as a Maclaurin series (power series about the origin) w.r.t to r. Doing this we get terms like

    \begin{equation}
        \label{eq:series}
        g(t, r) = 1 + tr + \left(\frac{3}{2}t^2 - \frac{1}{2}\right)r^2 + \left(\frac{5}{2}t^3 - \frac{3}{2}t  \right)r^3 + \left( \frac{35}{8}t^4 - \frac{15}{4}t^2 + \frac{3}{8} \right)r^4 + \mathcal{O}(r^5)
    \end{equation}

    The coefficents of the powers of $r$ are polynomials in $t$. The order of the polynomials matches the order of $r$ in each term and thus if we denote the polynomial of order $l$ in front of $r^l$ by
    $P_l(t)$, we get for the series expansion:

    \begin{equation}
        g(t, r) = \sum_{l = 0}^{\infty}P_l(t)r^l
    \end{equation}

    Remembering that $t = \cos\theta$ we finally get a way of writing inverse distances in terms of a power series involving polynomials of the cosine of the angle between the position vectors:

    \begin{equation}
        \frac{1}{\left| \vtr{r} - \vtr{r}' \right|} = \sum_{l = 0}^{\infty}P_l(\cos\theta)r^l, \ \ \ \ \left| \vtr{r}' \right| = 1
    \end{equation}

    The polynomials $P_l(t)$ we have discovered are known as Legendre polynomials. In physical applications their domain is only the interval $[-1, 1]$ since $t = \cos\theta \in [-1, 1]$.
    So far we have only seen a few Legendre polynomials in the coefficients of the series expansion (\ref{eq:series}), and have yet to develop a general formula for describing them. Developing
    a general formula is important as it allows one to calculate the terms of the expansion much more quickly. There are many formulas describing the Legendre polynomials but since we are interested
    in solving the associated Legendre equation, let us take a look at how the ordinary Legendre equation relates to the Legendre polynomials and from that extend our reach to the associated Legendre
    equation.

    \subsubsection{Associated Legendre Equation Part 2: The Ordinary Legendre Equation}

    As previously discussed, terms involving the inverse distance between two points are very common in physics. By developing the inverse distance in spherical coordinates as a series, the Legendre
    polynomials resulted. Another common feature in equations describing physical systems is the Laplacian $\nabla^2$, which is \textendash \ naievly speaking \textendash \ a kind of dot product of
    the gradient operator nabla $\nabla$ with itself: $\nabla\cdot\nabla = \left( \pdv{}{x}, \pdv{}{y}, \pdv{}{z} \right)\cdot\left( \pdv{}{x}, \pdv{}{y}, \pdv{}{z} \right) = \pdv[order = 2]{}{x} +
    \pdv[order = 2]{}{y} + \pdv[order = 2]{}{z} \equiv \nabla^2$. Why is the Laplacian so common? It describes diffusion, the movement of anything from a region of higher concentration to that 
    of a lower concentration. This can be the diffusion of heat, particles or even in some sence the intensity of a gravitational or electric field since they in some way diffuse when moving further
    away from the objet eminating the fields. For example, the equation describing gravitational potential, Poissons equation for gravity, contains the laplacian:

    \begin{equation}
        \nabla^2\varphi = 4\pi G\rho
    \end{equation}

    where $\varphi$ is the gravitational potential, $G$ is the universal gravitational constant and $\rho$ is the mass density distribution. It is thus no surprise that there may come a situation
    when the laplacian is to be applied to the inverse distance:

    \begin{equation}
        \nabla^2\left(\frac{1}{\left| \vtr{r} - \vtr{r}' \right|}\right) = \ ?
    \end{equation}

    Using the representation of the Laplacian in spherical coordinates and the power series form of the inverse distance we get:

    \begin{align}
        \nabla^2\left(\frac{1}{\left| \vtr{r} - \vtr{r}' \right|}\right) &= \nabla^2\left( \sum_{l = 0}^{\infty}P_l(\cos\theta)r^{l} \right) \\
        &= \frac{1}{r^2}\pdv{}{r}\left( r^2\pdv{}{r}\left[ \sum_{l=0}^{\infty}P_l(\cos\theta)r^{l} \right] \right) + \frac{1}{r^2\sin\theta}\pdv{}{\theta}\left( \sin\theta\pdv{}{\theta}\left[ 
        \sum_{l=0}^{\infty}P_l(\cos\theta)r^l\right]\right) \\
        &\quad + \frac{1}{r^2\sin\theta}\pdv[order = 2]{}{\varphi}\left[ \sum_{l=0}^{\infty}P_l(\cos\theta)r^l \right] \\
        \intertext{As there is no $\varphi$-dependence in the series expansion, the last term vanishes. Carrying out the partial derivative w.r.t to $r$ simply lowers the exponent of each $r^l$ by one
        and brings out a factor of $l$ to the series while killing the zeroth order term, meaning the summation now starts from $l = 1$. At the same time the partial derivative w.r.t to theta brings
        out a factor of $-\sin\theta$ due to the chain rule:}
        &= \frac{1}{r^2}\pdv{}{r}\left(r^2\sum_{l=1}^{\infty}lP_l(\cos\theta)r^{l - 1}\right) + \frac{1}{r^2\sin\theta}\pdv{}{\theta}\left(\sin\theta\sum_{l=0}^{\infty}-\sin\theta P'_l(\cos\theta)r^l\right) \\
        \intertext{Multiplying $r^2$ and $\sin\theta$ into their respective series yields:}
        &= \frac{1}{r^2}\pdv{}{r}\left(\sum_{l=1}^{\infty}lP_l(\cos\theta)r^{l + 1}\right) + \frac{1}{r^2\sin\theta}\pdv{}{\theta}\left(\sum_{l=0}^{\infty}-\sin^2\theta P'_l(\cos\theta)r^l\right) \\
        \intertext{Performing the partial derivatives once more we get:}
        &= \frac{1}{r^2}\left(\sum_{l=1}^{\infty}l(l + 1)P_l(\cos\theta)r^{l}\right) \\
        &\quad + \frac{1}{r^2\sin\theta}\left(\sum_{l=0}^{\infty}\left[\sin^3\theta P''_l(\cos\theta) -2\sin\theta\cos\theta P'_l(\cos\theta) \right]r^l\right) \\
        \intertext{Multiplying $\frac{1}{r^2}$ and $\frac{1}{r^2\sin\theta}$ into their respective series yields:}
        &= \sum_{l=1}^{\infty}l(l + 1)P_l(\cos\theta)r^{l - 2} + \sum_{l=0}^{\infty}\left[\sin^2\theta P''_l(\cos\theta) -2\cos\theta P'_l(\cos\theta)  \right]r^{l - 2} \\
        \intertext{The starting value of $l$ can be set to 0 in the first series without affecting the result since the factor of $l(l+1)$ goes to zero when $l = 0$:}
        &= \sum_{l=0}^{\infty}l(l + 1)P_l(\cos\theta)r^{l - 2} + \sum_{l=0}^{\infty}\left[\sin^2\theta P''_l(\cos\theta) -2\cos\theta P'_l(\cos\theta)  \right]r^{l - 2} \\
        \intertext{Since the limits and exponents of $r$ match in each of the series they can be combined into a single series:}
        &= \sum_{l=0}^{\infty}\Big[l(l + 1)P_l(\cos\theta) + \sin^2\theta P''_l(\cos\theta) - 2\cos\theta P'_l(\cos\theta)  \Big]r^{l - 2} \\
        \intertext{Writing $\sin^2\theta$ in terms of cosine, susbtituting $t = \cos\theta$ and ordering the terms from higher order derivative to lower order derivative we get:}
        \nabla^2\left(\frac{1}{\left| \vtr{r} - \vtr{r}' \right|}\right) &= \sum_{l=0}^{\infty}\Big[(1 - t^2)P''_l(t) -2tP'_l(t) + l(l + 1)P_l(t)\Big]r^{l - 2}
    \end{align}  

    Thus, if we had a situation where we knew that the Laplacian of the inverse distance was zero, we would have the following differential equation:

    \begin{align}
        \nabla^2\left(\frac{1}{\left| \vtr{r} - \vtr{r}' \right|}\right) &= 0 \\
        \sum_{l=0}^{\infty}\Big[(1 - t^2)P''_l(t) -2tP'_l(t) + l(l + 1)P_l(t)\Big]r^{l - 2} &= 0 \\
        \intertext{A series being equal to zero is equivalent to all of its coefficients being separately equal to zero meaning we get:}
        (1 - t^2)P''_l(t) - 2tP'_l(t) + l(l + 1)P_l(t) &= 0
    \end{align}

    The last differential equation is equivalent to the Legendre equation (\ref{eq:legendre}). Since the functions in the last differential equation are the Legendre polynomials,
    it is clear that they are valid solutions to the Legendre equation provided that $l$ is a natural number. The Legendre polynomials are of course not the only possible solutions but
    they are of interest to us since in physics we require $l$ to be a natural number. We have thus found the functions that solve the ordinary Legendre equation. Because of the similarity
    between the associated and ordinary Legendre equations, it is tempting to wonder whether the Legendre polynomials have a role in the solution of the associated Legendre equation as well.
    Soon we will see that this intuition is true it, but first we need to relate the Legendre polynomials to the associated Legendre equation

    \subsubsection{Associated Legendre Equation Part 3: Deriving the Associated Legendre Equation}

    Determining the connection between the Legendre polynomials and the associated Legendre equation while not terribly difficult, involves a lot of tedious algebraic manipulation.
    There have been attempts to make this analysis more intuitive but to this day the way this analysis is covered is very similar to the analysis done by mathematcians and physicists
    in the 19th centrury. We start the analysis by recalling the legendre equation:

\begin{comment}
    We start the analysis by recalling the generating function of the Legendre polynomials:


    \begin{equation}
        g(t, r) = \frac{1}{\sqrt{1 + r^2 - 2rt}}
    \end{equation}

    Recall also how the generating function related to the Legendre polynomials $P_l(t)$:

    \begin{equation}
        g(t, r) = \sum_{l = 0}^{\infty}P_l(t)r^l
    \end{equation}

    By lack of any direction to take, let us play around with the generating function by differentiating it a number of times, say $m \in \mathbb{N}$, w.r.t to t:

    \begin{align}
        \pdv[order = m]{g}{t} &= \pdv[order = m]{}{t}\left[ \frac{1}{\sqrt{1 + r^2 - 2rt}} \right] \\
        \intertext{Writing the inverse square root as an expression raised to the $-\frac{1}{2}$ power we get:}
        \pdv[order = m]{g}{t} &= \pdv[order = m]{}{t}\left[ (1 + r^2 - 2rt)^{-1/2} \right] \\
        \intertext{We know that each differentiation lowers the order of the exponent by 1 and brings the current exponent to the front as a factor.
        Thus we know that after $m$ differentiations, the exponent will be $-\frac{2m + 1}{2}$. For one differentiation it comes out to $-\frac{3}{2}$, for two to $-\frac{5}{2}$, etc.
        In addition to that, by the chain rule, we get a factor of $-2r$ to the front. Thus the factors that come out front for each successive differentiation look like 
        $-\frac{1}{2}(-2r), -\frac{3}{2}(-2r), -\frac{5}{2}(-2r), \dots$, which reduce to $r, 3r, 5r, \dots$. Thus after $m$ differentiations we have a multiplication consisting of $m$ $r$:s and
        the numbers 1 to $2m - 1$ (for example for $m = 3$ we have $(1)(3)(5)r^3$, since $2m - 1 = 2(3) - 1 = 5$). This multiplication of odd numbers can be denoted using the double factorial $!!$.
        By definition the double factorial multiplies every second number together up to the number whose factorial is being taken so for example $5!! = 1\cdot3\cdot5$. Thus we get:}
        \pdv[order = m]{g}{t} &= (2m - 1)!!r^m(1 + r^2 - 2rt)^{-(2m + 1)/2}
        \intertext{We know: $-(2m + 1)/2 = -(m + 1/2)$. Writing the negative exponent term as an inverse we finally get for the m:th derivative:}
        \pdv[order = m]{g}{t} &= \frac{(2m - 1)!!r^m}{(1 + r^2 - 2rt)^{m + 1/2}}
    \end{align}

    Meanwhile, differentiating the series expansion of the generating function we get:

    \begin{align}
        \odv[order = m]{g}{t} &= \odv[order = m]{}{t}\left[\sum_{l = 0}^{\infty}P_l(t)r^l\right] \\
        \odv[order = m]{g}{t} &= \sum_{l = m}^{\infty}P^{(m)}_l(t)r^l
    \end{align}
    
    where $P^{(m)}_l(t)$ denotes differentiation w.r.t. to $t$ $m$ times. Note that the sum now starts from $l = m$ since all terms with an order less than
    $m$ have vanished after differentiating $m$ times.
\end{comment}

    \begin{equation}
        \label{eq:leg}
        (1 - t^2)\odv[order = 2]{f}{t} - 2t\odv{f}{t} + l(l + 1)f = 0
    \end{equation}

    The form of the associated Legendre equation (89) we derived earlier written in terms of $f(t)$ instead of $\Theta(u)$ is:

    \begin{equation}
        \label{eq:assoc}
        (1 - t^2)\odv[order = 2]{f}{t} - 2t\odv{f}{t} + \left( A - \frac{m_\ell^2}{1 - u^2} \right)f = 0
    \end{equation}

    It can be seen that (\ref{eq:assoc}) reduces to (\ref{eq:leg}) when $A = l(l + 1)$ and $m_\ell = 0$. From this is it tempting to declare the Legendre equation a special case of 
    the associated Legendre equation. This is in fact a valid declaration and let's us give a definite value to the constant $A$ by setting it equal to $l(l + 1)$. Since the Legendre
    equation is a special case of the associated Legendre equation, perhaps the solutions of the associated Legendre equation can be expressed in terms of Legendre polynomials. This
    turns out to be true, and the connection is differentiation. By differentiating the Legendre equation $m$ times one can derive the Associated Legendre equation where $m_\ell = m$.
    From the previous section we know the solutions of interest of the Legendre equation to be Legendre polynomials, ie. $f(t) = P_l(t)$ in (\ref{eq:leg}). Thus we have:

    \begin{equation}
        (1 - t^2)\odv[order = 2]{P_l(t)}{t} - 2t\odv{P_l(t)}{t} + l(l + 1)P_l(t) = 0
    \end{equation}

    Let us now take the derivative w.r.t to $t$ of the above equation a total of $m$ times, for natural number $m$. Clearly when $m = 0$ the equation doesn't change and we still
    have the Legendre equation. This is analogous to setting $m_\ell = 0$ in the associated Legendre equation and getting the Legendre equation out as a special case. In the same
    way then, differentiating $m$ times corresponds to the associated Legendre equation where $m_\ell = m$:

    \begin{equation}
        \odv[order = m]{}{t}\left[ (1 - t^2)\odv[order = 2]{P_l(t)}{t} \right] - \odv[order = m]{}{t}\left[ 2t\odv{P_l(t)}{t} \right] + l(l+1)\odv[order = m]{P_l(t)}{t} = 0
    \end{equation}

    According to the Leibniz product rule for differentiation, the $m$:th derivative of a product of functions can be written as a sum involving binomial coeffiecients in a form
    very analogous to the ordinary binomial theorem for $m$:th powers of a binomial:

    \begin{equation}
        (fg)^{(m)} = \sum_{k = 0}^{m}\binom{m}{k}f^{(m - k)}g^{(k)}
    \end{equation}

    Using this result we get:

    \begin{equation}
        \begin{aligned}
            \sum_{k = 0}^{m}\left\{\binom{m}{k}\left[\odv[order = m-k]{}{t}(1 - t^2)\right]\left[\odv[order = k]{}{t}\odv[order = 2]{P_l(t)}{t}\right]\right\}
            - \sum_{k = 0}^{m}\left\{\binom{m}{k}\left[\odv[order = m-k]{}{t}(2t)\right]\left[\odv[order = k]{}{t}\odv{P_l(t)}{t} \right]\right\}& \\
            + \ l(l+1)\odv[order = m]{P_l(t)}{t} &\, = 0
        \end{aligned}
    \end{equation}

    In the first sum, the orders of the derivatives of $(1 - t^2)$ range from $m$ to zero (= no derivative). Since the order of $(1 - t^2)$ is 2, it can only be differentiated
    twice before dissapperaring. Thus only the last three terms remain. meaning $k = m - 2,\, m - 1,\, m$. Similarly in the second sum, $2t$ can only be differentiated once before
    disappearing meaning onyle the last two terms remain. In the second sum $k = m - 1,\, m$. Thus we get:

    \begin{equation}
        \begin{aligned}
            \bigg\{\binom{m}{m - 2}\left[\odv[order = m - (m - 2)]{}{t}(1 - t^2)\right]\left[\odv[order = m - 2]{}{t}\odv[order = 2]{P_l(t)}{t}\right]& \\
                + \binom{m}{m - 1}\left[\odv[order = m - (m - 1)]{}{t}(1 - t^2)\right]\left[\odv[order = m - 1]{}{t}\odv[order = 2]{P_l(t)}{t}\right]
            + \binom{m}{m}\left[\odv[order = m - (m)]{}{t}(1 - t^2)\right]\left[\odv[order = m]{}{t}\odv[order = 2]{P_l(t)}{t}\right]\bigg\}& \\
            - \bigg\{\binom{m}{m - 1}\left[\odv[order = m - (m - 1)]{}{t}(2t)\right]\left[\odv[order = m - 1]{}{t}\odv{P_l(t)}{t} \right]
            + \binom{m}{m}\left[\odv[order = m - (m)]{}{t}(2t)\right]\left[\odv[order = m]{}{t}\odv{P_l(t)}{t} \right]\bigg\}& \\ 
            + \ l(l+1)\odv[order = m]{P_l(t)}{t} &\, = 0
        \end{aligned}
    \end{equation}

    For the binomial coefficients we get:

    \begin{equation}
        \binom{m}{m - 2} = \frac{m!}{(m - 2)!(m - (m - 2))!} = \frac{m!}{(m - 2)!2! = \frac{1}{2}m(m - 1)}
    \end{equation}

    \begin{equation}
        \binom{m}{m - 1} = \frac{m!}{(m - 1)!(m - (m - 1))!} = \frac{m!}{(m - 1)!1!} = m
    \end{equation}

    \begin{equation}
        \binom{m}{m} = \frac{m!}{m!(m - (m))!} = \frac{1}{0!} = \frac{1}{1} = 1
    \end{equation}

    Thus we have:

    \begin{equation}
        \begin{aligned}
            \bigg\{\frac{1}{2}m(m - 1)\left[\odv[order = 2]{}{t}(1 - t^2)\right]\left[\odv[order = m - 2 + 2]{P_l(t)}{t}\right]& \\
                + m\left[\odv[order = 1]{}{t}(1 - t^2)\right]\left[\odv[order = m - 1 + 2]{P_l(t)}{t}\right]
            + \left[\odv[order = 0]{}{t}(1 - t^2)\right]\left[\odv[order = m + 2]{P_l(t)}{t}\right]\bigg\}& \\
            - m\left[\odv[order = 1]{}{t}(2t)\right]\left[\odv[order = m - 1 + 1]{P_l(t)}{t} \right]
            + \left[\odv[order = 0]{}{t}(2t)\right]\left[\odv[order = m + 1]{P_l(t)}{t} \right]\bigg\}& \\ 
            + \ l(l+1)\odv[order = m]{P_l(t)}{t} &\, = 0
        \end{aligned}
    \end{equation}

    $\odv[order = 0]{}{t}$ denotes that no derivative is taken. Simplifying further we get:

    \begin{equation}
        \begin{aligned}
            \bigg\{\frac{1}{2}m(m - 1)\left[-2\right]\left[\odv[order = m]{P_l(t)}{t}\right]& \\
                + m\left[-2t\right]\left[\odv[order = m + 1]{P_l(t)}{t}\right]
            + \left[1 - t^2\right]\left[\odv[order = m + 2]{P_l(t)}{t}\right]\bigg\}& \\
            - m\left[2\right]\left[\odv[order = m]{P_l(t)}{t} \right]
            + \left[2t\right]\left[\odv[order = m + 1]{P_l(t)}{t} \right]\bigg\}& \\ 
            + \ l(l+1)\odv[order = m]{P_l(t)}{t} &\, = 0
        \end{aligned}
    \end{equation}

    Upon denoting $\odv[order = m]{P_l(t)}{t}$ more compactly as $P^{(m)}_l(t)$ we get:

    \begin{equation}
        \begin{aligned}
            \left\{-m(m - 1)P^{(m)}_l(t) + -2mt\odv{P^{(m)}_l(t)}{t} + (1 - t^2)\odv[order = 2]{P^{(m)}_l(t)}{t}\right\}& \\
            - \left\{2mP^{(m)}_l(t) + 2t\odv{P^{(m)}_l(t)}{t}\right\} + l(l+1)P^{(m)}_l(t) &\, = 0
        \end{aligned}
    \end{equation}

    Collecting terms of equal order we get:

    \begin{equation}
        (1 - t^2)\odv[order = 2]{P^{(m)}_l(t)}{t} - 2(m + 1)t\odv{P^{(m)}_l(t)}{t} + \Big[ l(l + 1) - m(m - 1) - 2m \Big]P^{(m)}_l(t) = 0
    \end{equation}

    \begin{equation}
        \label{eq:close}
        (1 - t^2)\odv[order = 2]{P^{(m)}_l(t)}{t} - 2t(m + 1)\odv{P^{(m)}_l(t)}{t} + \Big[ l(l + 1) - m(m + 1)\Big]P^{(m)}_l(t) = 0
    \end{equation}

    Equation (\ref{eq:close}) is starting to look similar to the associated Legendre equation but we not quite there yet. One important thing to notice is that
    while the ordinary Legendre equation was self-adjoint, equation (\ref{eq:close}) is not. This is a problem since the Hamiltonian in quantum mechanics is always
    self-adjoint and consequently any equations we derive from it must also be self adjoint. The framework of Sturm\textendash Liouville theory tells us that by multiplying
    a general second order linear ODE by a suitable integrating factor one can bring the equation to a self-adjoint form. This is great news for us but is contains a caveat
    which will become clear shortly. The general form of a Sturm\textendash Liouville problem is:

    \begin{equation}
        \label{eq:sturm-liouville}
        \odv{}{x}\left[ p(x)\odv{f}{x} \right] + q(x)f = -\lambda w(x)f
    \end{equation}

    For some functions $p(x), q(x)$ and a weight function $w(x)$ as well as an eigenvalue $\lambda$. Moreover, given a general second order linear ODE of the form

    \begin{equation}
        \label{eq:general}
        P(x)f'' + Q(x)f' + R(x)f = 0
    \end{equation}

    it can be shown that the by multiplying (\ref{eq:general}) by the integrating factor

    \begin{equation}
        \mu(x) = \exp\left(\int^{x} \frac{Q(u) - P'(u)}{P(u)} \odif{u}\right)
    \end{equation}

    the equation can be brought into self-adjoint Sturm\textendash Liouville form. The caveat of this is that multiplying the equation by the integrating factor $\mu(x)$ changes
    the weight function $w(x)$ from $w(x)$ to $\mu(x)w(x)$. This means that the orthogonality relation of the eigenfunctions $f_{\lambda}$ of the Sturm\textendash Liouville changes
    since in general the orthogonality relation depends on $w(x)$:

    \begin{equation}
        \langle f_n | f_m \rangle = \int_{a}^{b}f_n(x)f_m(x)w(x)\odif{x} = \delta_{nm}
    \end{equation}

    For the Legendre equation, the weight function is simply $w(x) = 1$, since $\lambda = l(l + 1)$ and consequently it should also be $1$ for the associated Legendre equation since no factor appears in 
    front of $l(l+1)$ in that equation either. Thus care must be taken to restore the weight function back to $1$ after multiplying equation (\ref{eq:close}) by the integrating factor that brings
    it to self-adjoint form. Let us now proceed with determining the integrating factor. From (\ref{eq:close}) we can make the identifications:

    \begin{equation}
        P(t) = 1 - t^2, \ \ \ \ Q(t) = -2(m + 1)t, \ \ \ \ R(t) = l(l + 1) - m(m + 1)
    \end{equation}

    From this we get for the integrating factor $\mu(t)$:

    \begin{align}
        \mu(t) &= \exp\left( \int^{t} \frac{Q(u) - P'(u)}{P(u)} \odif{u} \right) \\
        \mu(t) &= \exp\left( \int^{t} \frac{-2(m + 1)u - (-2u)}{1 - u^2} \odif{u} \right) \\
        \mu(t) &= \exp\left(m\int^{t} \frac{-2u}{1 - u^2} \odif{u} \right) \\
        \intertext{The integrand is a product of the form $f'(g(u))g'(u)$, where $g(u) = 1 - u^2$, $g'(u) = -2u$, and $f'(g) = \frac{1}{g}$. Therefore the integration
        can be performed straightforwardly by applying the chain rule in reverse. We know that $\odv{}{x}\ln x = \frac{1}{x}$, so we get:}
        \mu(t) &= \exp\left(m \ln|1 - u^2|\Big|^{t} \right) \\
        \mu(t) &= \exp\left(m \ln|1 - t^2| \right) \\
        \mu(t) &= \mathrm{e}^{m \ln|1 - t^2|} \\
        \mu(t) &= \left(\mathrm{e}^{\ln|1 - t^2|}\right)^{m} \\
        \mu(t) &= |1 - t^2|^{m} \\
        \intertext{Since $t = \cos\theta \in [-1, 1] \implies t^2 \in [0, 1]$, we conclude that $1 - t^2$ is always nonnegative and the absolute value signs can be removed:}
        \mu(t) &= (1 - t^2)^{m}
    \end{align}

    Let us now multiply (\ref{eq:close}) by $\mu(t)$:

    \begin{equation}
        (1 - t^2)^{m}(1 - t^2)\odv[order = 2]{P^{(m)}_l}{t} - (1 - t^2)^{m}2t(m + 1)\odv{P^{(m)}_l}{t} + (1 - t^2)^{m}\Big[ l(l + 1) - m(m + 1)\Big]P^{(m)}_l = 0
    \end{equation}

    \begin{equation}
        (1 - t^2)^{m + 1}\odv[order = 2]{P^{(m)}_l}{t} - 2t(m + 1)(1 - t^2)^{m}\odv{P^{(m)}_l}{t} + (1 - t^2)^{m}\Big[ l(l + 1) - m(m + 1)\Big]P^{(m)}_l = 0
    \end{equation}

    Recognizing that $-2t(m + 1)(1 - t^2)^{m} = \odv{}{t}\left( 1 - t^2 \right)^{m + 1}$ we can write:
    
    \begin{equation}
        \odv{}{t}\left[ (1 - x^2)^{m + 1}\odv{P^{(m)}_{l}}{t} \right] + (1 - t^2)^{m}\Big[ l(l + 1) - m(m + 1)\Big]P^{(m)}_l = 0
    \end{equation}

    Finally arranging the equation to match the form of (\ref{eq:sturm-liouville}) we get:

    \begin{equation}
        \label{eq:almost}
        \odv{}{t}\left[ (1 - t^2)^{m + 1}\odv{P^{(m)}_{l}}{t} \right] -m(m + 1)(1 - t^2)^{m}P^{(m)}_l = -l(l + 1)(1 - t^2)^{m}P^{(m)}_{l}
    \end{equation}

    We can now clearly see that the weight function is $w(t) = (1 - t^2)^{m}$. We want the weight function to be equal to unity so we must perform a change of variables
    to accomplish that. Because of [INSERT INTUITION HERE] let us perform a change of variables, where we define a new function $P^{m}_{l}(t)$. Take care to note that
    $P^{m}_l(t) \neq P^{(m)}_{l}(t)$ meaning that this is not a relabelling of an already existing variable but instead an introdction of a new (albiet similarly named) variable.
    Let $P^{m}_{l}(t) = (1 - t^2)^{m/2}P^{(m)}_{l}(t)$. Thus our original functions expressed in terms of the new function are $P^{(m)}_{l}(t) = (1 - t^2)^{-m/2}P^{m}_l(t)$. For convenience,
    let us derive an expression for $\odv{P^{(m)}_{l}(t)}{t}$ in terms of $P^{m}_{l}(t)$:

    \begin{align}
        \odv{P^{(m)}_{l}}{t} &= \odv{}{t}\left[ (1 - t^2)^{-m/2}P^{m}_{l} \right] \\
        \intertext{By the product rule we get:}
        \odv{P^{(m)}_{l}}{t} &= (1 - t^2)^{-m/2}\odv{P^{m}_{l}}{t} + \left( -\frac{m}{2} \right)(1 - t^2)^{-m/2 - 1}(-2t)P^{m}_{l} \\
        \odv{P^{(m)}_{l}}{t} &= (1 - t^2)^{-m/2}\odv{P^{m}_{l}}{t} + (1 - t^2)^{-m/2}\frac{mt}{1 - t^2}P^{m}_{l} \\
        \intertext{Factoring out $(1 - t^2)^{-m/2}$ we get:}
        \odv{P^{(m)}_{l}}{t} &= (1 - t^2)^{-m/2}\left[\odv{P^{m}_{l}}{t} + \frac{mt}{1 - t^2}P^{m}_{l}\right]
    \end{align}
    
    Plugging the result above and $P^{(m)}_{l}(t) = (1 - t^2)^{-m/2}P^{m}_{l}(t)$ into (\ref{eq:almost}) we get:

    \begin{equation}
        \begin{aligned}
            \odv{}{t}\left[ (1 - t^2)^{m + 1}\left( (1 - t^2)^{-m/2}\left[\odv{P^{m}_{l}}{t} + \frac{mt}{1 - t^2}P^{m}_{l}\right] \right)\right]& \\
            - m(m + 1)(1 - t^2)^{m}(1 - t^2)^{-m/2}P^{m}_{l} &= -l(l + 1)(1 - t^2)^{m}(1 - t^2)^{-m/2}P^{m}_{l}
        \end{aligned}
    \end{equation}

    The exponents simplify: $m + 1 - \frac{m}{2} = \frac{2m + 2 - m}{2} = \frac{m + 2}{2} = \frac{m}{2} + 1$ and $m - \frac{m}{2} = \frac{m}{2}$:

    \begin{equation}
        \odv{}{t}\left[ (1 - t^2)^{m/2 + 1}\left(\odv{P^{m}_{l}}{t} + \frac{mt}{1 - t^2}P^{m}_{l}\right)\right]
        - m(m + 1)(1 - t^2)^{m/2}P^{m}_{l} = -l(l + 1)(1 - t^2)^{m/2}P^{m}_{l}
    \end{equation}

    Multiplying $(1 - t^2)$ into the parentheses we get:

    \begin{equation}
        \odv{}{t}\left[ (1 - t^2)^{m/2}\left((1 - t^2)\odv{P^{m}_{l}}{t} + mtP^{m}_{l}\right)\right]
        - m(m + 1)(1 - t^2)^{m/2}P^{m}_{l} = -l(l + 1)(1 - t^2)^{m/2}P^{m}_{l}
    \end{equation}

    Applying the product rule we get:  

    \begin{equation}
        \begin{aligned}
            (1 - t^2)^{m/2}\left[(1 - t^2)\odv[order = 2]{P^{m}_{l}}{t} - 2t\odv{P^{m}_{l}}{t} + mt\odv{P^{m}_{l}}{t} + mP^{m}_{l}\right]& \\
            + \left(\frac{m}{2} \right)(-2t)(1 - t^2)^{m/2 - 1}\left[(1 - t^2)\odv{P^{m}_{l}}{t} + mtP^{m}_{l}\right]& \\
            - m(m + 1)(1 - t^2)^{m/2}P^{m}_{l} &= -l(l + 1)(1 - t^2)^{m/2}P^{m}_{l}
        \end{aligned}
    \end{equation}

    Multiplying $(1 - t^2)^{-1}$ into the brackets on the second line we get:

    \begin{equation}
        \begin{aligned}
            (1 - t^2)^{m/2}\left[(1 - t^2)\odv[order = 2]{P^{m}_{l}}{t} - 2t\odv{P^{m}_{l}}{t} + mt\odv{P^{m}_{l}}{t} + mP^{m}_{l}\right]& \\
            - mt(1 - t^2)^{m/2}\left[\odv{P^{m}_{l}}{t} + \frac{mt}{1 - t^2}P^{m}_{l}\right]& \\
            - m(m + 1)(1 - t^2)^{m/2}P^{m}_{l} &= -l(l + 1)(1 - t^2)^{m/2}P^{m}_{l}
        \end{aligned}
    \end{equation}

    $(1 - t^2)^{m/2}$ is a common factor in all terms of the equation meaning it can be eliminated provided that it doesn't equal zero. $(1 - t^2)^{m/2} = 0 \iff t = \pm1$, meaning
    that we can proceed with the elimination since $t = \cos\theta \in [-1, 1]$. We can now notice that the weight function has been reduced back to unity:

    \begin{equation}
        \begin{aligned}
            \left[(1 - t^2)\odv[order = 2]{P^{m}_{l}}{t} - 2t\odv{P^{m}_{l}}{t} + mt\odv{P^{m}_{l}}{t} + mP^{m}_{l}\right]
            - mt\left[\odv{P^{m}_{l}}{t} + \frac{mt}{1 - t^2}P^{m}_{l}\right]& \\
            - m(m + 1)P^{m}_{l} &= -l(l + 1)P^{m}_{l}
        \end{aligned}
    \end{equation}

    Multiplying the term $-mt$ into the second brackets we get:

    \begin{equation}
        \begin{aligned}
            (1 - t^2)\odv[order = 2]{P^{m}_{l}}{t} - 2t\odv{P^{m}_{l}}{t} + \cancel{mt\odv{P^{m}_{l}}{t}} + mP^{m}_{l}
            - \cancel{mt\odv{P^{m}_{l}}{t}} - \frac{m^2t^2}{1 - t^2}P^{m}_{l}& \\
            - m(m + 1)P^{m}_{l} &= -l(l + 1)P^{m}_{l}
        \end{aligned}
    \end{equation}

    Putting the terms $mP^{m}_{l}$, \ $-\frac{m^2t^2}{1 - t^2}P^{m}_{l}$ \ and \ $-m(m+1)P^{m}_{l}$ under a common denominator we get:

    \begin{equation}
        (1 - t^2)\odv[order = 2]{P^{m}_{l}}{t} - 2t\odv{P^{m}_{l}}{t} + \frac{m(1 - t^2) - m^2t^2 - m(m + 1)(1 - t^2)}{1 - t^2}P^{m}_{l} = -l(l + 1)P^{m}_{l}
    \end{equation}

    Taking out $(1 - t^2)$ as a common factor from two of the terms in the fraction we get:

    \begin{align}
        (1 - t^2)\odv[order = 2]{P^{m}_{l}}{t} - 2t\odv{P^{m}_{l}}{t} + \frac{(1 - t^2)(m - m(m + 1)) - m^2t^2}{1 - t^2}P^{m}_{l} &= -l(l + 1)P^{m}_{l} \\
        (1 - t^2)\odv[order = 2]{P^{m}_{l}}{t} - 2t\odv{P^{m}_{l}}{t} + \frac{(1 - t^2)(m - m^2 - m)) - m^2t^2}{1 - t^2}P^{m}_{l} &= -l(l + 1)P^{m}_{l} \\
        (1 - t^2)\odv[order = 2]{P^{m}_{l}}{t} - 2t\odv{P^{m}_{l}}{t} + \frac{(1 - t^2)(-m^2) - m^2t^2}{1 - t^2}P^{m}_{l} &= -l(l + 1)P^{m}_{l} \\
        (1 - t^2)\odv[order = 2]{P^{m}_{l}}{t} - 2t\odv{P^{m}_{l}}{t} + \frac{-m^2 + \cancel{m^2t^2} - \cancel{m^2t^2}}{1 - t^2}P^{m}_{l} &= -l(l + 1)P^{m}_{l} \\
        (1 - t^2)\odv[order = 2]{P^{m}_{l}}{t} - 2t\odv{P^{m}_{l}}{t} + \frac{-m^2}{1 - t^2}P^{m}_{l} &= -l(l + 1)P^{m}_{l}
    \end{align}

    Lastly, by moving all the terms to the left side of the equation and taking $P^{m}_{l}$ as a common factor we finally get the associated Legendre equation:

    \begin{equation}
        \label{eq:finally}
        (1 - t^2)\odv[order = 2]{P^{m}_{l}}{t} - 2t\odv{P^{m}_{l}}{t} + \left(l(l + 1) - \frac{m^2}{1 - t^2}\right)P^{m}_{l} = 0
    \end{equation}

    \subsubsection{Associated Legendre Equation Part 4: The Solution}

    Now that we have derived the associated Legendre equation we are able to solve it. In fact, we have already derived the solution in the last section since we started the section
    from the Legendre equation and through simple but tedious manipulations we arrived at the associated Legendre equation. This means that the solutions to (\ref{eq:finally}) can be
    expressed in terms of Legendre polynomials. To derive the associated Legendre equation we first differentiated $m$ times and then performed a change of variables. Thus the original
    Legendre polynomials have been mapped from $P_{l}(t)$ to $(1 - t^2)^{m/2}\odv[order = m]{P_{l}(t)}{t} \equiv P^{m}_{l}(t)$. Thus we have found a family of solutions to the associated
    Legendre equation parametrized by $l$ and a new index $m$:

    \begin{equation}
        P^{m}_{l} = (1 - t^2)^{m/2}\odv[order = m]{P_{l}(t)}{t}
    \end{equation}

    Just like with the ordinary Legendre equation, these are not the only solutions that exist for the associated Legendre equation but are of interest to us because they are physically
    viable. The solutions are called the associated Legendre polynomials (even though they are not polynomials for odd $m$ since they contain rational exponents) and when parametrired with
    the cosine of the polar angle $\theta$, they solve the colalitude equation up to some constant $C_{\theta}$. Thus we have as the solution to the colatitude equation:

    \begin{equation}
        \label{eq:colatitude}
        \boxed{\Theta(\theta) = C_{\theta}P^{m}_{l}(\cos\theta)}
    \end{equation}

    Some things to note about the associated Legendre polynomials are that for each value of $l$, $m$ can range from $-l$ to $l$. This follows from the fact the legendre polynomial $P_{l}$ has
    order $l$ and can be differentiated at most $l$ times before disappearing. Since the associated Legendre polynomials are constructed by differentiating the ordinary Legendre polynomials this
    requirement becomes easy to understand. Another thing to note is that $m$ can have negative values (it ranges from $-l$ to $l$). This might seem counterintuitive at first since the definition
    of $P^{m}_{l}$ requires taking the $m$:th derivative and negative derivatives are not defined. The reason why negative values of $m$ can be allowed stems from the fact that the associated Legendre
    equation only depends on the square of $m$ meaning that the sign of $m$ does not matter to the equation. The way to define associated Legendre polynomials of negative order is through Rodrigues'
    formula for the Legendre Polyonimals:

    \begin{equation}
        \label{eq:rodrigues}
        P_{l}(t) = \frac{1}{2^{l}l!}\odv[order = l]{}{t}\left[ (1 - t^2)^{l} \right]
    \end{equation}

    With (\ref{eq:rodrigues}) it can be shown that $P^{m}_{l}$ for negative $m$ can be defined as a proportionality constant multiplied by $P^{m}_{l}$ with positive $m$:

    \begin{equation}
        P^{-m}_{l}(t) = (-1)^{m}\frac{(l - m)!}{(l + m)!}P^{m}_{l}(t)
    \end{equation}
    
    \subsection{The Radial Equation}
        
    The radial equation is of the form

    \begin{equation}
        \label{eq:radial}
        \odv{}{r}\left( r^2\odv{R}{r} \right) + \frac{2\mu r^2}{\hbar^2}\left( E + \frac{e^2}{4\pi\varepsilon_{0}r} \right)R - l(l + 1)R = 0
    \end{equation}

    where we have plugged in $A = l(l + 1)$ from the solution of the colatitude equation. This equation is another relatively difficult one to solve but with some effort
    a solution can be derived. Let us first reduce the radial equation to a form that is nicer to work with by performing the change of variables $u = Rr \iff R = \frac{u}{r}$.

    The derivative $\odv{R}{r}$ changes to:

    \begin{align}
        \odv{R}{r} &= \odv{(u/r)}{r} \\ 
        \odv{R}{r} &= \odv{}{r}\left( \frac{u}{r} \right) \\ 
        \odv{R}{r} &= u\left( -\frac{1}{r^2} \right) + \left( \odv{u}{r} \right)\frac{1}{r} \\ 
        \odv{R}{r} &= -\frac{u}{r^2} + \frac{1}{r}\odv{u}{r}
    \end{align}

    Plugging this into (\ref{eq:radial}) we get:

    \begin{align}
        \odv{}{r}\left( r^2\left( -\frac{u}{r^2} + \frac{1}{r}\odv{u}{r} \right)\right) + \frac{2\mu r^2}{\hbar^2}\left( E + \frac{e^2}{4\pi\varepsilon_{0}r} \right)\frac{u}{r} - l(l + 1)\frac{u}{r} &= 0 \\
        \odv{}{r}\left(-u + r\odv{u}{r} \right) + \frac{2\mu r^2}{\hbar^2}\left( E + \frac{e^2}{4\pi\varepsilon_{0}r} \right)\frac{u}{r} - l(l + 1)\frac{u}{r} &= 0 \\
        -\cancel{\odv{u}{r}} + r\odv[order = 2]{u}{r} + \cancel{\odv{u}{r}} + \frac{2\mu r^2}{\hbar^2}\left( E + \frac{e^2}{4\pi\varepsilon_{0}r} \right)\frac{u}{r} - l(l + 1)\frac{u}{r} &= 0 \\
        r\odv[order = 2]{u}{r} + \frac{2\mu r^2}{\hbar^2}\left( E + \frac{e^2}{4\pi\varepsilon_{0}r} \right)\frac{u}{r} - l(l + 1)\frac{u}{r} &= 0 \\
        \odv[order = 2]{u}{r} + \frac{2\mu r}{\hbar^2}\left( E + \frac{e^2}{4\pi\varepsilon_{0}r} \right)\frac{u}{r} - l(l + 1)\frac{u}{r^2} &= 0 \\
        \intertext{Multiplying $\frac{2\mu r}{\hbar^2}$ into the parentheses and moving the $E$ term to the other side of the equation we get:}
        \odv[order = 2]{u}{r} + \frac{2\mu e^2}{4\pi\varepsilon_{0}\hbar^2}\frac{u}{r} - l(l + 1)\frac{u}{r^2} &= -\frac{2\mu}{\hbar^2}Eu \\
        \intertext{Dividing out the factor $-\frac{2\mu}{\hbar^2}$ in front of the $E$ term we get the following eigenvalue equation:}
        -\frac{\hbar^2}{2\mu}\odv[order = 2]{u}{r} - \frac{e^2}{4\pi\varepsilon_{0}}\frac{u}{r} + \frac{l(l + 1)\hbar^2}{2\mu r^2}u &= Eu \\
        \intertext{Finally taking out $u$ as a common factor we get:}
        -\frac{\hbar^2}{2\mu}\odv[order = 2]{u}{r} + \left( -\frac{e^2}{4\pi\varepsilon_{0}r} + \frac{l(l + 1)\hbar^2}{2\mu r^2}\right)u &= Eu
    \end{align}

    Let us next make some general arguments about the form of $u(r)$. Since $u$ describes the radial part of the wavefunction of the electron we known that it must act as a probability distribution.
    This means that integrating the magnitude squared of $u$ over all space must result in the probability coming out to $1$. Because of this it seems intuitive that not only must $u$ be limited as
    $r \to \infty$, but that $u$ must tend to $0$ as we move further and further away from the origin. If this was not the case, the probability would diverge and this would not be a physically viable
    solution. Moreover since $u = rR$ we can see that $u(0) = 0\cdot R = 0$. We have thus made the following two observations:

    \begin{equation}
        \label{eq:bounds}
        \lim_{r\to \infty}u(r) = 0, \ \ \ \ u(0) = 0 
    \end{equation}

    Let us now reduce the equation into an ever simpler form by defining the following auxilliary variable $\kappa = \frac{2\sqrt{-2\mu E}}{\hbar}$. Since the electron in a hydrogen atom is in a bound
    state we know that $E < 0$ and therefore $-E = |E|$. Thus we get $\kappa = \frac{2\sqrt{2\mu |E|}}{\hbar}$. Dividing the last equation by $E$ we get:

    \begin{align}
        -\frac{\hbar^2}{2\mu E}\odv[order = 2]{u}{r} + \frac{1}{E}\left( -\frac{e^2}{4\pi\varepsilon_{0}r} + \frac{l(l + 1)\hbar^2}{2\mu r^2}\right)u &= u \\
        \frac{\hbar^2}{2\mu |E|}\odv[order = 2]{u}{r} + \left( -\frac{e^2}{4\pi\varepsilon_{0}Er} + \frac{l(l + 1)\hbar^2}{2\mu Er^2}\right)u &= u \\
        \frac{4}{\kappa^2}\odv[order = 2]{u}{r} + \left( \frac{e^2}{4\pi\varepsilon_{0}|E|r} - \frac{l(l + 1)\hbar^2}{2\mu |E|r^2}\right)u &= u \\
        \frac{4}{\kappa^2}\odv[order = 2]{u}{r} + \left( \frac{e^2}{4\pi\varepsilon_{0}|E|r} - \frac{4l(l + 1)}{\kappa^2r^2}\right)u &= u \\
        \intertext{Multiplying the term with the remaining $E$ by $1 = \frac{2\mu}{\hbar^2}\frac{\hbar^2}{2\mu}$ we get:}
        \frac{4}{\kappa^2}\odv[order = 2]{u}{r} + \left( \frac{e^2}{4\pi\varepsilon_{0}|E|r}\frac{2\mu}{\hbar^2}\frac{\hbar^2}{2\mu} - \frac{4l(l + 1)}{\kappa^2r^2}\right)u &= u \\
        \frac{4}{\kappa^2}\odv[order = 2]{u}{r} + \left( \frac{\mu e^2}{2\pi\varepsilon_{0}\hbar^2 r}\frac{\hbar^2}{2\mu|E|} - \frac{4l(l + 1)}{(\kappa r)^2}\right)u &= u \\
        \frac{4}{\kappa^2}\odv[order = 2]{u}{r} + \left( \frac{\mu e^2}{2\pi\varepsilon_{0}\hbar^2 r}\frac{4}{\kappa^2} - \frac{4l(l + 1)}{(\kappa r)^2}\right)u &= u \\
        \frac{4}{\kappa^2}\odv[order = 2]{u}{r} + \left( \frac{\mu e^2}{2\pi\varepsilon_{0}\hbar^2\kappa}\frac{4}{\kappa r} - \frac{4l(l + 1)}{(\kappa r)^2}\right)u &= u \\
        \intertext{Dividing the equation by $4$ we get:}
        \frac{1}{\kappa^2}\odv[order = 2]{u}{r} + \left( \frac{\mu e^2}{2\pi\varepsilon_{0}\hbar^2\kappa}\frac{1}{\kappa r} - \frac{l(l + 1)}{(\kappa r)^2}\right)u &= \frac{1}{4}u \\
        \intertext{Finally moving all the terms that multiply $u$'s to the right side of the equation we get:}
        \frac{1}{\kappa^2}\odv[order = 2]{u}{r} &= \left(\frac{1}{4} - \frac{\mu e^2}{2\pi\varepsilon_{0}\hbar^2\kappa}\frac{1}{\kappa r} + \frac{l(l + 1)}{(\kappa r)^2}\right)u
    \end{align}

    Now setting $\rho = \kappa r$ and $\rho_{0} = \frac{\mu e^2}{2 \pi \varepsilon_{0}\hbar^2 \kappa}$ the derivative becomes:

    \begin{align}
        \odv[order = 2]{u}{r} &= \odv{}{r}\left( \odv{u}{r} \right) \\
        &= \odv{}{r}\left( \odv{u}{\rho}\odv{\rho}{r} \right) \\
        &= \odv{}{r}\left( \odv{u}{\rho}\odv{(\kappa r)}{r} \right) \\
        &= \odv{}{r}\left( \kappa\odv{u}{\rho} \right) \\
        &= \kappa\odv{\odv{u}{\rho}}{r} \\
        &= \kappa\odv{\odv{u}{\rho}}{\rho}\odv{\rho}{r} \\
        &= \kappa\odv[order = 2]{u}{\rho}\odv{(\kappa r)}{r} \\
        \odv[order = 2]{u}{r} &= \kappa^2\odv[order = 2]{u}{\rho}
    \end{align}

    The equation becomes:

    \begin{align}
        \frac{1}{\kappa^2}\kappa^2\odv[order = 2]{u}{\rho} &= \left(\frac{1}{4} - \frac{\rho_{0}}{\rho} + \frac{l(l + 1)}{\rho^2}\right)u \\
        \odv[order = 2]{u}{\rho} &= \left(\frac{1}{4} - \frac{\rho_{0}}{\rho} + \frac{l(l + 1)}{\rho^2}\right)u
    \end{align}

    With the equation expressed in this form we can make some progress towards its solution. Since the radial distance cannot be negative the domain of this equation is $]0, \infty[$. Since this is 
    a linear differential equation we know by the uniqueness and existence theorem that if $\left( \frac{1}{4} - \frac{\rho_{0}}{\rho} + \frac{l(l+1)}{\rho^2} \right)$ is continuous in $]0, \infty[$
    and we have two boundary conditions for the equation, then the equation must have a unique solution. It it easy to see that $\left( \frac{1}{4} - \frac{\rho_{0}}{\rho} + \frac{l(l+1)}{\rho^2} \right)$
    is continuous in $]0, \infty[$ since $\frac{1}{\rho}$ and $\frac{1}{\rho^2}$ don't have any discontinuities there. We also have already found two boundary conditions for the solution (\ref{eq:bounds})
    which means that all of the conditions are met and we can conclude that if we find a solution for this equation it will be unique. In its current form, the equation we are studying does not
    lend itself to be solved easily. Let us instead study the asymptotic behaviour of this equation to maybe factor out some characteristics of the full solution from the equation.
    When $\rho \to \infty$, the terms with $\frac{1}{\rho}$ and $\frac{1}{\rho^2}$ dependence become very small and the differential equation reduces to:

    \begin{align}
        \odv[order = 2]{u}{\rho} &= \left( \frac{1}{4} - 0 + 0 \right)u \\ 
        \odv[order = 2]{u}{\rho} &= \frac{1}{4}u 
    \end{align}

    This equation shows that the second derivative of $u$ is equal to $\frac{1}{4}u$. This means that the solution must involve the exponential function. Since the equation is second order, the
    complete solution must be a linear combination of two linearly independent solutions. Because differentiating an exponential of the form $e^{ax}$ produces $ae^{ax}$, we need the exponentials
    to have a form where differentiating them twice leads to a factor of $\frac{1}{4}$ in front of the function. Thus it is easy to see that the general solution
    $u_{\infty}$ of the asymptotic equation is of the form:

    \begin{equation}
        u_{\infty}(\rho) = Ae^{\rho/2} + Be^{-\rho/2}
    \end{equation}

    We know from earlier analysis of the boundary conditions of $u$ (\ref{eq:bounds}) that $\lim_{r \to \infty}u(r) = 0$ and since $\rho \propto r$ we also know that $\lim_{\rho \to \infty}u(\rho) = 0$.
    This means that the term involving the positive exponent $Ae^{\rho/2}$ must vanish for the solution to agree with the boundary conditions since only the term with the negative exponent $Be^{-\rho/2}$
    goes to zero as $\rho$ approaches infinity. Thus we can reason that $A = 0$ and we get for the solution of the asymptotic equation:

    \begin{equation}
        u_{\infty}(\rho) = Be^{-\rho/2}
    \end{equation}

    Let us next consider the behaviour of the radial equation when $\rho \to 0$. The term with $\frac{1}{\rho^2}$ dependence grows much quicker than the term with only $\frac{1}{\rho}$ dependence
    (not to mention the constant term $\frac{1}{4}$ which doesn't grow at all) when $\rho$ is very close to zero and thus we can approximate the equation extremely near the boundary $\rho = 0$ as:

    \begin{align}
        \odv[order = 2]{u}{\rho} &= \left( 0 - 0 + \frac{l(l + 1)}{\rho^2} \right)u \\
        \odv[order = 2]{u}{\rho} &= \frac{l(l + 1)}{\rho^2}u
    \end{align}

    Let us multiply both sides of the equation by $\rho^2$:

    \begin{equation}
        \rho^2 \odv[order = 2]{u}{\rho} = l(l + 1)u
    \end{equation}

    Now the equation can be solved with a change of variables $t = \ln\rho \iff \rho e^{t}$. The derivative becomes:

    \begin{align}
        \odv[order = 2]{u}{\rho} &= \odv{}{\rho}\left( \odv{u}{\rho} \right) \\
        &= \odv{}{\rho}\left( \odv{u}{t}\odv{t}{\rho} \right) \\
        &= \odv{}{\rho}\left( \odv{u}{t}\odv{(ln \rho)}{\rho} \right) \\
        &= \odv{}{\rho}\left( \odv{u}{t}\frac{1}{\rho} \right) \\
        &= \odv{u}{t} \odv{}{\rho}\left(\frac{1}{\rho} \right) + \frac{1}{\rho}\odv{\odv{u}{t}}{\rho} \\
        &= -\frac{1}{\rho^2}\odv{u}{t} + \frac{1}{\rho}\odv{\odv{u}{t}}{t}\odv{t}{\rho} \\
        &= -\frac{1}{\rho^2}\odv{u}{t} + \frac{1}{\rho}\odv[order = 2]{u}{t}\odv{(\ln \rho)}{\rho} \\
        &= -\frac{1}{\rho^2}\odv{u}{t} + \frac{1}{\rho}\odv[order = 2]{u}{t}\frac{1}{\rho} \\
        \odv[order = 2]{u}{\rho} &= \frac{1}{\rho^2}\left(\odv[order = 2]{u}{t} - \odv{u}{t}\right)
    \end{align}

    The equation now becomes:

    \begin{align}
        \rho^2\frac{1}{\rho^2}\left( \odv[order = 2]{u}{t} - \odv{u}{t} \right) &= l(l + 1)u \\
        \odv[order = 2]{u}{t} - \odv{u}{t} - l(l + 1)u &= 0
    \end{align}

    This is a linear second order differential equation with constant coefficients. Thus the solution $u_0$ of the asymptotic equation is a linear combination of two linearly independent exponentials

    \begin{equation}
        u_{0}(t) = Ce^{\lambda_1 t} + De^{\lambda_2 t}
    \end{equation}

    where $\lambda_1$ and $\lambda_2$ are roots of the equation's characteristic polynomial $\lambda^2 - \lambda - l(l + 1)$:

    \begin{align}
        \lambda^2 - \lambda - l(l + 1) &= 0 \\
        \lambda &= \frac{1 \pm \sqrt{ 1 + 4l(l + 1)}}{2} \\
        \lambda &= \frac{1 \pm \sqrt{ 1 + 4l^2 + 4l}}{2} \\
        \lambda &= \frac{1 \pm \sqrt{(2l + 1)^2}}{2} \\
        \lambda &= \frac{1 \pm 2l + 1}{2} \\
        \lambda = \frac{1 + 2l + 1}{2} &\lor \lambda = \frac{1 - 2l - 1}{2} \\
        \lambda = \frac{2 + 2l}{2} &\lor \lambda = \frac{-2l}{2} \\
        \lambda = l + 1 &\lor \lambda = -l
    \end{align}

    Thus we have $\lambda_1 = l + 1$ and $\lambda_2 = -l$ and the solution becomes:

    \begin{equation}
        u_{0}(t) = Ce^{(l+1)t} + De^{-lt}
    \end{equation}

    Substituting back $e^{t} = \rho$ we get:

    \begin{equation}
        u_{0}(\rho) = C\rho^{l + 1} + D\rho^{-l}
    \end{equation}

    Again form the earlier analysis of the boundary conditions of $u$ (\ref{eq:bounds}) we know that $u(r=0) = 0$. Since $\rho \propto r$ we also know that $u(\rho=0) = 0$. This means that
    the term involving the negative exponent $D\rho^{-l}$ (remembering that $l \geq 0$) must vanish for the solution to agree with the boundary conditions. This because negative powers grow
    without bound as the argument goes to zero. Meanwhile the term with the positive exponent $C\rho^{l + 1}$ goes to zero as $\rho \to 0$ meaning that it accurately matches the boundary condition.
    Thus $D = 0$ and we get for the solution:

    \begin{equation}
        u_{0}(\rho) = C\rho^{l + 1}
    \end{equation}

    Now that we have two solutions that accurately match the boundary conditions of $u$, we can combine them into a solution $u_{0,\infty}$ that matches both boundary conditions by multiplying
    the solutions together:

    \begin{equation}
       u_{0,\infty}(\rho) = u_{0}(\rho)u_{\infty}(\rho) = BC\rho^{l + 1}e^{-\rho/2} = C\rho^{l + 1}e^{-\rho/2}
    \end{equation}

    This works because when $\rho = 0$, $u_{0} = 0$ meaning the whole function goes to zero and when $\rho \to \infty$, the exponential term in $u_{\infty}$ decays much quicker than the power term
    in $u_{0}$ meaning the boundary condition of $u$ going to zero when $\rho$ approaches infinity is also satisfied. It is tempting to think that this is the complete soluttion but thus far
    we have only defined how the function must behave near the boundaries where certain terms of the equation dominate the others. When we are far from the boundaries, all of the terms in the equation
    must be taken into into account meaning that the most honest thing we can do is write our current best solution $u$ as the product of $u_{0,\infty}$ and some other function $v$ of $\rho$:

    \begin{equation}
        u(\rho) = u_{0,\infty}(\rho)v(\rho) = C\rho^{l + 1}e^{-\rho/2}v(\rho)
    \end{equation}

    Let us substitute $u = u_{0,\infty}v$ into the radial equation. We will ignore the constant $C$ for now as we can reintroduce a constant later:

    \begin{align}
        \odv[order = 2]{}{\rho}\left( \rho^{l + 1}e^{-\rho/2}v \right) &= \left(\frac{1}{4} - \frac{\rho_{0}}{\rho} + \frac{l(l + 1)}{\rho^2}\right)(\rho^{l + 1}e^{-\rho/2}v) \\
        \odv{}{\rho}\left( \odv{}{\rho}(\rho^{l + 1})e^{-\rho/2}v + \rho^{l + 1}\odv{}{\rho}(e^{-\rho/2})v + \rho^{l + 1}e^{-\rho/2}\odv{}{\rho}(v)\right) &=
        \left(\frac{1}{4}\rho^{l+1} - \frac{\rho_{0}\rho^{l+1}}{\rho} + \frac{l(l + 1)\rho^{l+1}}{\rho^2}\right)e^{-\rho/2}v \\
        \odv{}{\rho}\left((l + 1)\rho^{l}e^{-\rho/2}v -\frac{1}{2}\rho^{l + 1}e^{-\rho/2}v + \rho^{l + 1}e^{-\rho/2}\odv{v}{\rho}\right) &=
        \left(\frac{1}{4}\rho^{l+1} - \rho_{0}\rho^{l} + l(l + 1)\rho^{l-1}\right)e^{-\rho/2}v \\
        \odv{}{\rho}\left(e^{-\rho/2}\left[(l + 1)\rho^{l}v -\frac{1}{2}\rho^{l + 1}v + \rho^{l + 1}\odv{v}{\rho}\right]\right) &=
        \left(\frac{1}{4}\rho^{l+1} - \rho_{0}\rho^{l} + l(l + 1)\rho^{l-1}\right)e^{-\rho/2}v \\[1em]
        e^{-\rho/2}\odv{}{\rho}\left[(l + 1)\rho^{l}v -\frac{1}{2}\rho^{l + 1}v + \rho^{l + 1}\odv{v}{\rho}\right]& \\
        + \odv{}{\rho}\left( e^{-\rho/2} \right)\left[ (l + 1)\rho^{l}v -\frac{1}{2}\rho^{l + 1}v + \rho^{l + 1}\odv{v}{\rho} \right] &= \left(\frac{1}{4}\rho^{l+1} - \rho_{0}\rho^{l}
        + l(l + 1)\rho^{l-1}\right)e^{-\rho/2}v
    \end{align}

    \begin{align}
        &e^{-\rho/2}\left[(l + 1)\odv{}{\rho}(\rho^{l})v + (l + 1)\rho^{l}\odv{}{\rho}(v) - \frac{1}{2}\odv{}{\rho}(\rho^{l + 1})v - \frac{1}{2}\rho^{l+1}\odv{}{\rho}v +
        \odv{}{\rho}(\rho^{l + 1})\odv{v}{\rho} + \rho^{l+1}\odv[order = 2]{v}{\rho}\right] \\
        &-\frac{1}{2}e^{-\rho/2}\left[ (l + 1)\rho^{l}v -\frac{1}{2}\rho^{l + 1}v + \rho^{l + 1}\odv{v}{\rho} \right] = \left(\frac{1}{4}\rho^{l+1} - \rho_{0}\rho^{l} + l(l + 1)\rho^{l-1}\right)e^{-\rho/2}v \\
        \intertext{Dividing out the common factor $e^{-\rho/2}$ we get:}
        &l(l + 1)\rho^{l-1}v + (l + 1)\rho^{l}\odv{v}{\rho} - \frac{1}{2}(l + 1)\rho^{l}v - \frac{1}{2}\rho^{l+1}\odv{v}{\rho} +
        (l + 1)\rho^{l}\odv{v}{\rho} + \rho^{l+1}\odv[order = 2]{v}{\rho} \\
        &-\frac{1}{2}(l + 1)\rho^{l}v + \frac{1}{4}\rho^{l + 1}v - \frac{1}{2}\rho^{l + 1}\odv{v}{\rho} = \frac{1}{4}\rho^{l+1}v - \rho_{0}\rho^{l}v + l(l + 1)\rho^{l-1}v
    \end{align}
     
    The terms $l(l+1)\rho^{l - 1}v$ and $\frac{1}{4}\rho^{l + 1}v$ can be found on both sides of the equation and thus vanish:

    \begin{equation}
        (l + 1)\rho^{l}\odv{v}{\rho} - \frac{1}{2}(l + 1)\rho^{l}v - \frac{1}{2}\rho^{l+1}\odv{v}{\rho} + (l + 1)\rho^{l}\odv{v}{\rho} + \rho^{l+1}\odv[order = 2]{v}{\rho}
        -\frac{1}{2}(l + 1)\rho^{l}v - \frac{1}{2}\rho^{l + 1}\odv{v}{\rho} = - \rho_{0}\rho^{l}v
    \end{equation}

    Apart from the term $\rho^{l + 1}\odv[order = 2]{v}{\rho}$, each term on the left hand side of the equation appears twice:

    \begin{align}
        \rho^{l + 1}\odv[order = 2]{v}{\rho} + 2(l + 1)\rho^{l}\odv{v}{\rho} - (l + 1)\rho^{l}v - \rho^{l+1}\odv{v}{\rho} &= - \rho_{0}\rho^{l}v \\
        \intertext{Moving all terms to the left hand side and taking some common factors results in:}
        \rho^{l + 1}\odv[order = 2]{v}{\rho} + (2(l + 1) - \rho)\rho^{l}\odv{v}{\rho} + [\rho_{0} - (l + 1)]\rho^{l}v &= 0 \\
        \intertext{Dividing the whole equation by $\rho^{l}$ we get:}
        \rho\odv[order = 2]{v}{\rho} + (2l + 2 - \rho)\odv{v}{\rho} + (\rho_{0} - l - 1)v &= 0
        \intertext{Finally, separating the $2$ inside the first parentheses into $1 + 1$ we get:}
        \rho\odv[order = 2]{v}{\rho} + ((2l + 1) + 1 - \rho)\odv{v}{\rho} + (\rho_{0} - l - 1)v &= 0
    \end{align}

    The last equation encountered is known as the associated Laguerre equation. The general form of the associated laguerre equation is:

    \begin{equation}
        x\odv[order = 2]{f}{x} + (\alpha + 1 - x)\odv{f}{x} + nf = 0, \ \ \ \ \alpha, n \in \mathbb{N}
    \end{equation}

    The fact that $\alpha$ and $n$ are whole numbers is only a requirement if we want the solutions to be polynomials. Moreover the physics of the situation constrain $\alpha$ and $n$ to be
    whole numbers so we continue with that assumption. From the general associated Laguerre equation we can make the identifications $\alpha = 2l + 1$ and $n = \rho_{0} - l - 1$.
    The associated Laguerre equation could be solved by first deriving it from the ordinary Laguerre equation by differentiation $\alpha$ times similarly to how the associated Legendre equation
    was solved earlier. We'll instead solve it using a series solution since it provides us with a closed form expression for the associated Laguerre polynomials. Using the series ansatz
    $v(\rho) = \sum_{k = 0}^{\infty}c_k\rho^{k + t}$ let us, for convinience, calculate the first and second derivative of the series:

    \begin{align}
        \odv{v}{\rho} &= \odv{}{\rho}\left( \sum_{k = 0}^{\infty}c_k \rho^{k + t} \right) \\
        \odv{v}{\rho} &= \sum_{k = 0}^{\infty}c_k \odv{}{\rho}(\rho^{k + t}) \\
        \odv{v}{\rho} &= \sum_{k = 0}^{\infty}c_k (k + t)\rho^{k + t - 1}
    \end{align}

    \begin{align}
        \odv[order = 2]{v}{\rho} &= \odv{}{\rho}\left( \odv{v}{\rho} \right) \\
        \odv[order = 2]{v}{\rho} &= \odv{}{\rho}\left(\sum_{k = 0}^{\infty}c_k (k + t)\rho^{k + t - 1}  \right) \\
        \odv[order = 2]{v}{\rho} &= \sum_{k = 0}^{\infty}c_k (k + t)\odv{}{\rho}(\rho^{k + t - 1}) \\
        \odv[order = 2]{v}{\rho} &= \sum_{k = 0}^{\infty}c_k (k + t)(k + t - 1)\rho^{k + t - 2} 
    \end{align}
    
    Now plugging in the series ansatz and its derivatives into the associated Laguerre equation we get:

    \begin{align}
        \rho\left(\sum_{k = 0}^{\infty}c_k (k + t)(k + t - 1)\rho^{k + t - 2}\right) + ((2l + 1) + 1 - \rho)\left(\sum_{k = 0}^{\infty}c_k (k + t)\rho^{k + t - 1}\right)& \\
        + (\rho_{0} - l - 1)\left(\sum_{k = 0}^{\infty}c_k\rho^{k + t}\right) &= 0 \\
        \intertext{Multiplying the coefficients in front of the series into the series we get:}
        \left(\sum_{k = 0}^{\infty}c_k(k + t)(k + t - 1)\rho^{k + t - 1}\right) + \left(\sum_{k = 0}^{\infty}c_k((2l + 1) + 1)(k + t)\rho^{k + t - 1}\right)& \\
        - \left( \sum_{k=0}^{\infty}c_k(k + t)\rho^{k + t}\right) + \left(\sum_{k = 0}^{\infty}c_k(\rho_{0} - l - 1)\rho^{k + t}\right) &= 0 \\
        \intertext{The extra series appeared when we multiplied $((2l + 1) + 1 - \rho) = ((2l + 1) + 1) - \rho$ separately into the series. Let us now comibne all series where the powers of $\rho$ are equal:}
        \left(\sum_{k = 0}^{\infty}c_k\left[(k + t)(k + t - 1) + ((2l + 1) + 1)(k + t)\right]\rho^{k + t - 1}\right)& \\
        + \left( \sum_{k=0}^{\infty}c_k\left[(\rho_{0} - l - 1) - (k + t)\right]\rho^{k + t}\right) &= 0 \\
        \left(\sum_{k = 0}^{\infty}c_k\left[(k + t)(k + t + (2l + 1))\right]\rho^{k + t - 1}\right)& \\
        + \left( \sum_{k=0}^{\infty}c_k(\rho_{0} - l - 1 - k - t)\rho^{k + t}\right) &= 0 \\
        \intertext{Let us take the first term of the first series outside the series:}
        c_0\left[(0 + t)(0 + t + (2l + 1))\right]\rho^{0 + t - 1} + \left(\sum_{k = 1}^{\infty}c_k\left[(k + t)(k + t + (2l + 1))\right]\rho^{k + t - 1}\right)& \\
        + \left( \sum_{k=0}^{\infty}c_k(\rho_{0} - l - 1 - k - t)\rho^{k + t}\right) &= 0 \\
        \intertext{In the first sum, let us shift the index $k$ to $j + 1$ and in the second sum we'll simply relabel all $k$:s to $j$:s:}
        c_0t(t + (2l + 1))\rho^{t - 1} + \left(\sum_{j = 0}^{\infty}c_{j+1}\left[(j + t + 1)(j + t + 1 + (2l + 1))\right]\rho^{j + t}\right)& \\
        + \left( \sum_{j=0}^{\infty}c_j(\rho_{0} - l - 1 - j - t)\rho^{j + t}\right) &= 0
    \end{align}

    Now the powers of $\rho$ match in both series and they can be combined:
    
    \begin{equation}
        c_0t(t + (2l + 1))\rho^{t - 1} + \left(\sum_{j = 0}^{\infty}\Big\{c_{j+1}\left[(j + t + 1)(j + t + 1 + (2l + 1))\right] + c_j(\rho_{0} - l - 1 - j - t)\Big\}\rho^{j + t}\right) = 0 \\
    \end{equation}

    When the series on the left is equated to the zero-series on the right we know that by the uniqueness of pwer series all the coefficients must be separately equal to zero. Therefore we have
    the following conditions:

    \begin{equation}
        c_0t(t + (2l + 1)) = 0 \ \ \ \ \land \ \ \ \ c_{j+1}\left[(j + t + 1)(j + t + 1 + (2l + 1))\right] + c_j(\rho_{0} - l - 1 - j - t) = 0
    \end{equation}

    The first condition is known as the indical equation and its solutions are $t_1 = 0$ and $t_2 = -(2l + 1)$. Note that $c_0 = 0$, while possible is not a meaningful solution since the 
    coefficients $c_{j+1}$ are proportional to the previous coefficient $c_j$ and thus if $c_0 = 0$, then $c_1 = kc_0 = k\cdot0 = 0$, $c_2 = kc_1 = k\cdot0 = 0$ and so on meaning we only
    get the trivial solution of $v = 0$. If instead we use the roots of the indical equation $t_1$ and $t_2$ we meaningful results. Since we have develpoed the series about the point $\rho = 0$, which
    is a regular singularity of the Laguerre equation and the difference between the roots of the indical equation $t_1 - t-2 = 0 - (-(2l + 1)) = 2l + 1$ is a nonzero integer we know that
    the series method will only give us one polynomial solution and that the other linearly independent solutions contains a logarithmic term. Since we are only interested in the polynomial solution
    (we don't need another linearly independent function since we already have the $\rho^{l+1}$ and $e^{-\rho/2}$ terms to satisty the boundary conditions), we can proceed by setting $t = t_1 = 0$ which
    will produce the polynomial solution. Thus the second requirement becomes:

    \begin{align}
        c_{j + 1}\left[(j + 1)(j + 1 + (2l + 1))\right] + c_j(\rho_{0} - l - 1 - j) &= 0 \\
        \intertext{For convenience and analogously to the general associated Legendre equation let us denote $2l + 1 = \alpha$ and $\rho_{0} - l - 1 = n$:}
        c_{j + 1}\left[(j + 1)(j + 1 + \alpha)\right] + c_j(n - j) &= 0
    \end{align}
    
    Solving this in terms of $c_{j + 1}$ we get a recursion relation for the coefficients of the power series:

    \begin{align}
        c_{j + 1} &= \frac{-(n - j)}{(j + 1)(j + 1 + \alpha)}c_j \\
        c_{j + 1} &= \frac{(j - n)}{(j + 1)(j + 1 + \alpha)}c_j
    \end{align}

    Let us change the index $j + 1 \to k$:

    \begin{equation}
        \label{eq:index}
        c_{k} = \frac{(k - n - 1)}{k(k + \alpha)}c_{k - 1}
    \end{equation}

    Before proceeding with the recursive formula, let us study the asymptotic behaviour of the coefficients $c_k$. For very large values of $k$ the other terms
    become negligible and we can write:

    \begin{align}
        c_{k} &\sim \frac{k}{k^2}c_{k - 1} \\
        c_{k} &\sim \frac{1}{k}c_{k - 1} \\
        \intertext{Iterating this formula until $c_0$ we get:}
        c_{k} &\sim \frac{1}{k\cdot(k - 1)\cdot(k - 2)\cdot\dots\cdot2\cdot1}c_0 \\
        c_{k} &\sim \frac{1}{k!}c_0 \\
        \intertext{This means that the whole series is of the form:}
        v(\rho) &\sim c_0\sum_{k = 0}^{\infty}\frac{1}{k!}\rho^{k} \\
        \intertext{We recognize this as the Taylor series for the exponential function:}
        v(\rho) &\sim c_0e^{\rho}
    \end{align}
    
    If this was the solution, we would have $u(\rho) = \rho^{l + 1}e^{-\rho/2}c_0e^{\rho} = c_0\rho^{l + 1}e^{\rho/2}$. This would cause $u$ to diverge as $\rho\to\infty$ and would make the wavefunction
    unphysical as it would no longer result in a finite probability. Therefore we conclude that the series expansion must terminate at some finite index $k_{\mathrm{max}}$ so that the exponential
    decay term $e^{-\rho/2}$ can keep the asymptotic behaviour of the function in check. From (\ref{eq:index}) we can see that if we want the $k$:th term to go to zero, $\frac{(k - n)}{k(k + \alpha)}$
    must go to zero. This can happen only if $k = n = \rho_{0} - l - 1$, meaning that our final accepted solution will be a polynomial of order $k_{\mathrm{max}} = \rho_{0} - l - 1$. This also forces
    $\rho_{0}$ to be an integer since $k$ and $l$ are integers and thus $\rho_{0} = k + l + 1$ can only be an integer. This is the source of our final quantum number $n$ which we will get to later. For
    now we'll keep denoting $\rho_{0}$ as $\rho_{0}$. Let us now continue where we left off with equation (\ref{eq:index}). Using the recursive formula, we can write $c_{k - 1}$ in terms of $c_{k - 2}$:

    \begin{align}
        c_{k} &= \frac{(k - n - 1)}{k(k + \alpha)}\frac{(k - n - 2)}{(k - 1)(k + \alpha - 1)}c_{k - 2} \\
        \intertext{Let us continue this until we have expressed $c_k$ in terms of $c_0$:}
        c_{k} &= \frac{(k - n - 1)}{k(k + \alpha)}\frac{(k - n - 2)}{(k - 1)(k + \alpha - 1)}\frac{(k - n - 3)}{(k - 2)(k + \alpha - 2)}\dots\frac{(k - n - k)}{1(k + \alpha - (k - 1))}c_0 \\
        c_{k} &= \frac{(k - n - 1)}{k(k + \alpha)}\frac{(k - n - 2)}{(k - 1)(k + \alpha - 1)}\frac{(k - n - 3)}{(k - 2)(k + \alpha - 2)}\dots\frac{(-n)}{1(\alpha + 1)}c_0 \\
        \intertext{Let us take out a minus sign from each of the terms in the numerators. Since there are $k$ terms in the product, the total sign will be $(-1)^k$:}
        c_{k} &= (-1)^k\frac{(n + 1 - k)}{k(k + \alpha)}\frac{(n + 2 - k)}{(k - 1)(k + \alpha - 1)}\frac{(n + 3 - k)}{(k - 2)(k + \alpha - 2)}\dots\frac{n}{1(\alpha + 1)}c_0 \\
        \intertext{The terms $k, k-1, k-2, \dots, 1$ in the denominator produce a factorial $k!$ whereas the terms $k + \alpha, k + \alpha - 1, k + \alpha - 2 \dots \alpha + 1$ don't produce a factorial
        since their product stops at $(\alpha + 1)$ instead of 1. We can still continue the factorial all the way to the end if we remember to divide by the missing part of the factorial, which
        is clearly $\alpha!$. Finally, the terms $n + 1 - k, n + 2 - k, n + 3 - k, \dots n$ in the numerator are going up, meaning that the product in the numerator can be written as the
        factorial of $n$. (the highest term in the product) divided by the missing part, which is $(n - k)!$. Thus we get:}
        c_{k} &= (-1)^k\frac{n!}{(n - k)!}\frac{1}{k!\frac{(k + \alpha)!}{\alpha!}}c_0 \\
        c_{k} &= (-1)^k\frac{n!}{(n - k)!}\frac{\alpha!}{k!(k + \alpha)!}c_0
    \end{align}

    Now all that remains undetermined is the initial coefficient $c_0$. Since it is just a constant we could just leave it undefined and incorporate it into a generirc constant in the front of our
    function $u(\rho)$ but since the recursion relation mathces exactly that of the associated Legendre polynomials which are an established family of polynomials, it would make sense to express
    our solution in terms of those as well. For the associated legendre polynomials the standard defines $c_0 = \binom{n + \alpha}{n}$ and so will we:

    \begin{align}
        c_{k} &= (-1)^k\frac{n!}{(n - k)!}\frac{\alpha!}{k!(k + \alpha)!}\binom{n + \alpha}{n} \\
        \intertext{Expanding the (generalized) binomial coefficient $\binom{n + \alpha}{n}$ we get:}
        c_{k} &= (-1)^k\frac{n!}{(n - k)!}\frac{\alpha!}{k!(k + \alpha)!}\frac{(n + \alpha)!}{n!(n + \alpha - n)!} \\
        c_{k} &= (-1)^k\frac{n!}{(n - k)!}\frac{\alpha!}{k!(k + \alpha)!}\frac{(n + \alpha)!}{n!\alpha!} \\
        c_{k} &= (-1)^k\frac{1}{(n - k)!}\frac{1}{k!(k + \alpha)!}(n + \alpha)! \\
        \intertext{Reorganizing the terms we get:}
        c_{k} &= \frac{(-1)^k}{k!}\frac{(n + \alpha)!}{(n - k)!(k + \alpha)!} \\
        \intertext{Recognizing that $k + \alpha = n + \alpha - n + k = n + \alpha - (n - k)$ we get:}
        c_{k} &= \frac{(-1)^k}{k!}\frac{(n + \alpha)!}{(n - k)!(n + \alpha - (n - k))!} \\
        \intertext{The expression in the second fraction corresponds to the (generalized) binomial coefficient $\binom{n + \alpha}{n - k}$ so we finally get:}
        c_{k} &= \frac{(-1)^k}{k!}\binom{n + \alpha}{n - k} \\
        \intertext{Recalling that $\alpha = 2l + 1$ and $n = \rho_{0} - l - 1$ we get:}
        c_{k} &= \frac{(-1)^k}{k!}\binom{\rho_{0} - l - 1 + 2l + 1}{\rho_{0} - l - 1 - k} \\
        c_{k} &= \frac{(-1)^k}{k!}\binom{\rho_{0} + l}{\rho_{0} - l - 1 - k} \\
    \end{align}

    Now our final solution for $v(\rho)$ looks like:

    \begin{align}
        v(\rho) = \sum_{k = 0}^{k_{\mathrm{max}}}\frac{(-1)^k}{k!}\binom{\rho_{0} + l}{\rho_{0} - l - 1 - k}\rho^{k + t} \\
        \intertext{Recalling that $k_{\mathrm{max}} = \rho_{0} - l - 1$ and $t = 0$ as well as the fact that $\rho_{0}$ was determined to be an integer, which corresponds to the principal quantum
        number $n$ (This $n$ is not the same $n$ that we used before as an auxiliary variable) so $\rho_{0} \to n$ we get:}
        v(\rho) = \sum_{k = 0}^{n - l - 1}\frac{(-1)^k}{k!}\binom{n + l}{n - l - 1 - k}\rho^{k}
    \end{align}

    Now we can finally write the solution to the radial equation:

    \begin{align}
        u(\rho) &= e^{-\rho/2}\rho^{l+1}\sum_{k = 0}^{n - l - 1}\frac{(-1)^k}{k!}\binom{n + l}{n - l - 1 - k}\rho^{k} \\
        \intertext{Recalling that the sum was constructed to correspond exactly to the associated Laguerre polynomials $L^{(\alpha)}_{n}$, where $\alpha = 2l + 1$ and $n = \rho - l - 1$, where we
        renamed $\rho_{0}$ to $n$ we get:}
        u(\rho) &= e^{-\rho/2}\rho^{l+1}L^{(2l + 1)}_{n - l - 1}(\rho) \\
        \intertext{It is convenient to stay in the scaled coordinates $\rho = \kappa r$ since the length scales of the hydrogen atom are very small. Instead recalling that
        $R = \frac{u}{r} = u\frac{\kappa}{\rho}$ we get:}
        R(\rho) &= \frac{\kappa}{\rho}e^{-\rho/2}\rho^{l+1}L^{(2l + 1)}_{n - l - 1}(\rho) \\
        R(\rho) &= \kappa e^{-\rho/2}\rho^{l}L^{(2l + 1)}_{n - l - 1}(\rho)
    \end{align}

    Recalling that $\kappa = \frac{2\sqrt{2\mu|E|}}{\hbar}$ and that $\rho_{0} = \frac{\mu e^2}{2\pi\varepsilon_{0}\hbar^2\kappa} \iff \kappa = \frac{\mu e^{2}}{2\pi\varepsilon_{0}\hbar^2\rho_{0}}$
    we get an equation relating the energy with $\rho_{0}$:

    \begin{align}
        \frac{2\sqrt{2\mu |E|}}{\hbar} &= \frac{\mu e^{2}}{2\pi\varepsilon_{0}\hbar^2\rho_{0}} \\
        \intertext{Solving for $E$ we get:}
        \sqrt{|E|} &= \frac{\hbar}{2\sqrt{2\mu}}\frac{\mu e^{2}}{2\pi\varepsilon_{0}\hbar^2\rho_{0}} \\
        |E| &= \frac{\hbar^2}{4(2)\mu}\frac{\mu^2 e^{4}}{4\pi^2\varepsilon_{0}^2\hbar^4\rho_{0}^2} \\        
        |E| &= \frac{\mu e^{4}}{32\pi^2\varepsilon_{0}^2\hbar^2}\frac{1}{\rho_{0}^2} \\
        \intertext{Since $E < 0$, $|E| = -E$. Also again setting $\rho_{0} = n$ we have:}
        E_n &= -\frac{\mu e^{4}}{32\pi^2\varepsilon_{0}^2\hbar^2}\frac{1}{n^2}
    \end{align}

    Thus we have derived an expression for the energy of the system. Next it is useful to express the $\kappa$ term in the expression for $R$ in terms of basic physical constants and $\rho_{0} = n$.
    A common way to do this is to use the Bohr radius $a_{0} = \frac{4\pi\varepsilon_{0}\hbar^2}{\mu e^2}$. Since $\kappa = \frac{\mu e^{2}}{2\pi\varepsilon_{0}\hbar^2 n}$,
    we can see that $\kappa = \frac{2}{na_0}$. This is the form of $\kappa$ we'll use going forward, but we actually don't need to write it down in our final expression for $R$ since we don't know
    the solution exactly but up to some constant $C_{\rho}$ which will be determined once we normalize the final wavefunction. Thus we can absorb the constant $\kappa$ into the arbitrary constant $C_{\rho}$
    and finally write the solution to the radial equation as:

    \begin{equation}
        \label{eq:radial-solution}
        \boxed{R(\rho) = C_{\rho}e^{-\rho/2}\rho^{l}L^{(2l + 1)}_{n - l - 1}(\rho), \ \ \ \ \rho = \kappa r = \frac{2r}{na_0}}
    \end{equation}

    Note that this isn't exactly the solution to the initial radial equation \ref{eq:radial} we set out to solve since we have solved it in terms of the variable $\rho$ instead of the variable $r$.
    Nevetheless this is the solution we want because the convention in quantum physics is to work with the scaled coordinates.

    \subsection{Putting it all together}

    We have now successfully found the solutions of the azimuthal, colatitude and radial equations up to some constants $C_{\varphi}$, $C_{\theta}$ and $C_{r}$. Because the original partial
    differential equation was separable, the complete solution to the equation is the product of the solutions we have found: $\psi = R\Theta\Phi$.

    \begin{equation}
        \psi(\rho, \theta, \varphi) = C_{\varphi}e^{im\varphi}C_{\theta}P^{m}_{l}( \cos\theta)C_{\rho}e^{-\rho/2}\rho^{l}L^{(2l + 1)}_{n - l - 1}(\rho)
    \end{equation}

    In order to make the solution fit into the formalism of quantum mechanics we must find the value of some constant $N$ such that when the solution is multiplied by $N$, it becomes normalized.
    This is necessary since for the solution to be a valid wave function the norm squared of the solution must behave like a probability distribution function. The constants $C_{\varphi}$, $C_{\theta}$
    and $C_{\rho}$ can be compressed into the mystery constant $N$ we wish to solve for. Since norm squared of a function is equivalent to the inner product of the function with itself we have the
    following requirement:

    \begin{equation}
        \braket{\psi}{\psi} = \braket{NR\Theta\Phi}{NR\Theta\Phi} = N^2\braket{R\Theta\Phi}{R\Theta\Phi} = 1
    \end{equation}

    The Hilbert space we are operating in is the space of square integrable functions (also known as $L^2$), which means that the inner product is defined as the integral over all space
    of the norm squared of the wavefunction:

    \begin{equation}
        \braket{\psi}{\psi} = \int_{\set{R}^3}|\psi|^2 J(\vtr{r})\odif[order=3]{\vtr{r}}
    \end{equation}

    Where $J(\vtr{r)}$ is the Jacobian determinant of whatever coordinate system we happen to be working in. Since the wavefunction is a complex number it's norm squared can be written as $\psi^{\ast}\psi$.
    Also recalling that the Jacobian for spherical coordinates is $r^2 \sin\theta$ and that the integration limits for each of the variables $r, \theta, \varphi$ are 0 to $\infty$, 0 to $\pi$ and 
    0 to $2\pi$ respectively we get:

    \begin{equation}
        \label{eq:inprod}
        \braket{\psi}{\psi} = \int_{0}^{2\pi}\int_{0}^{\pi}\int_{0}^{\infty}\psi^{\ast}\psi r^2 \sin\theta \odif{r, \theta, \varphi}        
    \end{equation}

    The wave function in expression \ref{eq:inprod} is expected to be a function of $r$, $\theta$ and $\varphi$ while the solution we have come up with $\psi = R\Theta\Phi$ is a function of
    $\rho$, $\theta$ and $\varphi$. This means that we need to perform one final change of variables in order to evaluate the inner product correctly. Recalling that $\rho = \kappa r$ and
    we get that $r = \frac{\rho}{\kappa}$ and $\odif{r} = \frac{1}{\kappa}\odif{\rho}$. The integration bounds don't change since $\rho = 0$ when $r = 0$ and $\rho \to \infty$ when $r \to \infty$.
    Performing the substitution we get:

    \begin{align}
        \braket{\psi}{\psi} &= \int_{0}^{2\pi}\int_{0}^{\pi}\int_{0}^{\infty}\psi^{\ast}\psi \left( \frac{\rho}{\kappa} \right)^2 \sin\theta \frac{1}{\kappa}\odif{\rho, \theta, \varphi} \\
        \intertext{Since $\kappa$ is a constant we can pull out the factor of $\frac{1}{\kappa^3}$ in front of the integral finally giving us an expression we can use to normalize the solution:}
        \braket{\psi}{\psi} &= \frac{1}{\kappa^3}\int_{0}^{2\pi}\int_{0}^{\pi}\int_{0}^{\infty}\psi^{\ast}\psi \rho^2 \sin\theta \odif{\rho, \theta, \varphi}
    \end{align}

    By setting the above integral multiplied by $N^2$ equal to 1 we can solve for the value of $N$ required to normalize our solution:

    \begin{align}
        1 &= N^2\frac{1}{\kappa^3}\int_{0}^{2\pi}\int_{0}^{\pi}\int_{0}^{\infty}\psi^{\ast}(\rho,\theta,\varphi)\psi(\rho,\theta,\varphi) \rho^2\sin\theta\odif{\rho,\theta,\varphi} \\
        \intertext{Plugging in $\psi = R\Theta\Phi$ we get:}
        1 &= N^2\frac{1}{\kappa^3}\int_{0}^{2\pi}\int_{0}^{\pi}\int_{0}^{\infty}R^{\ast}\Theta^{\ast}\Phi^{\ast}R\Theta\Phi \rho^2\sin\theta\odif{\rho,\theta,\varphi} \\
        \intertext{$R$ and $\Theta$ are both real and thus $R^{\ast} = R$ and $\Theta^{\ast} = \Theta$. $\Phi$ is a complex exponential $e^{im\varphi}$ and thus its complex conjugate is $e^{-im\varphi}$.
        When multiplied together we get $\Phi^{\ast}\Phi = e^{-im\varphi}e^{im\varphi} = e^{0} = 1$. Thus we have:}
        1 &= N^2\frac{1}{\kappa^3}\int_{0}^{2\pi}\int_{0}^{\pi}\int_{0}^{\infty}R^2\Theta^2\odif{\rho,\theta,\varphi} \\
        \intertext{Since $R$ is only a function of $\rho$ and $\Theta$ is only a function of $\theta$, the integrations can be performed separately:}
        1 &= N^2\frac{1}{\kappa^3}\int_{0}^{2\pi}\odif{\varphi}\int_{0}^{\pi}\Theta^2\sin\theta\odif{\theta}\int_{0}^{\infty}R^2\rho^2\odif{\rho} \\
        \intertext{The first integral yields $2\pi$. Plugging in the expressions for $\Theta$ and $R$ we get:}
        1 &= N^2\frac{2\pi}{\kappa^3}\int_{0}^{\pi}\left(P^{m}_{l}(\cos\theta)\right)^2\sin\theta\odif{\theta}\int_{0}^{\infty}\left(e^{-\rho/2}\rho^{l}L^{(2l + 1)}_{n - l - 1}(\rho)\right)^2\rho^2\odif{\rho} \\
        1 &= N^2\frac{2\pi}{\kappa^3}\int_{0}^{\pi}\left(P^{m}_{l}(\cos\theta)\right)^2\sin\theta\odif{\theta}\int_{0}^{\infty}e^{-\rho}\rho^{2l + 2}\left(L^{(2l + 1)}_{n - l - 1}(\rho)\right)^2\odif{\rho}
    \end{align}

    In order to evaluate the remaining two integrals, we turn to the following orthogonality relations for the associated Legendre polynomials and the associated Laguerre polynomials. These
    relations can be proven using the Rodrigues' formulas for each family of polynomials:

    \begin{equation}
        \label{eq:legendre-orto}
        \int_{-1}^{1}P^{m}_{k}(t)P^{m}_{l}(t)  \odif{t} = \frac{2}{2l + 1}\frac{(l + m)!}{(l - m)!}\delta_{kl}
    \end{equation}

    \begin{equation}
        \label{eq:laguerre-orto}
        \int_{0}^{\infty} e^{-\rho}\rho^{\alpha} L^{(\alpha)}_{m}(\rho)L^{(\alpha)}_{n}(\rho) \odif{\rho} = \frac{(n + \alpha)!}{n!}\delta_{nm}
    \end{equation}

    We can quickly see that the first integral can be transformed into (\ref{eq:legendre-orto}) by the substitution $t = \cos\theta$ and thus we can write:
    
    \begin{equation}
        1 = N^2\frac{2\pi}{\kappa^3}\frac{2}{2l + 1}\frac{(l + m)!}{(l - m)!}\int_{0}^{\infty}e^{-\rho}\rho^{2l + 2}\left(L^{(2l + 1)}_{n - l - 1}(\rho)\right)^2\odif{\rho}
    \end{equation}

    The remaining integral corresponds almost exactly to (\ref{eq:laguerre-orto}) except for the fact that the exponent of $\rho$ is $2l + 2$ instead of $2l + 1$. Luckily there is
    a formula that relates an associated Laguerre polynomial multiplied by the independent variable to Laguerre polynomials with the same order (ie. value of the $\alpha$ parameter):

    \begin{equation}
        \rho L^{(\alpha)}_{n}(\rho) = (\alpha + 2n + 1)L^{(\alpha)}_{n}(\rho) - \frac{n + 1}{\alpha + n + 1}L^{(\alpha)}_{n + 1}(\rho) - (\alpha + n)^2L^{(\alpha)}_{n - 1}(\rho)
    \end{equation}
        
    Thus multiplying a single $\rho$ to one of the Laguerre polynomials in the integral as to reduce the power $2l + 2 \to 2l + 1$ we get:

    \begin{align}
        \rho L^{(2l + 1)}_{n - l - 1}(\rho) &= (2l + 1 + 2(n - l - 1) + 1)L^{(2l + 1)}_{n - l - 1}(\rho) \\
        &\quad - \frac{n - l - 1 + 1}{2l + 1 + n - l - 1 + 1}L^{(2l + 1)}_{n - l - 1 + 1}(\rho) - (2l + 1 + n - l - 1)^2L^{(2l + 1)}_{n - l - 1 - 1}(\rho) \\[0.5em]
        \rho L^{(2l + 1)}_{n - l - 1}(\rho) &= 2nL^{(2l + 1)}_{n - l - 1}(\rho) - \frac{n - l}{n + l + 1}L^{(2l + 1)}_{n - l}(\rho) - (n + l)^2L^{(2l + 1)}_{n - l - 2}(\rho)
    \end{align}

    If we were to plug this into the integral, every term except the first one would be multiplying a Laguerre polynomial of a different degree. Since the Laguerre polynomials are orthogonal
    w.r.t to degree, we know that the integrals involving these terms go to zero. Thus it is sufficient for us to only consider the first term when evaluating the integral:

    \begin{equation}
        1 = N^2\frac{4\pi}{\kappa^3(2l + 1)}\frac{(l + m)!}{(l - m)!}\int_{0}^{\infty}e^{-\rho}\rho^{2l + 1}\left( 2nL^{(2l + 1)}_{n - l - 1}(\rho)\right)L^{(2l + 1)}_{n - l - 1}(\rho)\odif{\rho}
    \end{equation}

    When we take out the constant $2n$ outside the integral we have an integrand that is exactly of the form (\ref{eq:laguerre-orto}):

    \begin{equation}
        1 = N^2\frac{4\pi(2n)}{\kappa^3(2l + 1)}\frac{(l + m)!}{(l - m)!}\int_{0}^{\infty}e^{-\rho}\rho^{2l + 1}L^{(2l + 1)}_{n - l - 1}(\rho)L^{(2l + 1)}_{n - l - 1}(\rho)\odif{\rho}
    \end{equation}

    Therefore we get:

    \begin{align}
        1 &= N^2\frac{4\pi(2n)}{\kappa^3(2l + 1)}\frac{(l + m)!}{(l - m)!}\frac{[(n - l - 1) + (2l + 1)]!}{(n - l - 1)!} \\
        1 &= N^2\frac{4\pi(2n)}{\kappa^3(2l + 1)}\frac{(l + m)!}{(l - m)!}\frac{(n + l)!}{(n - l - 1)!} \\
        N^2 &= \frac{\kappa^3(2l + 1)(l - m)!(n - l - 1)!}{4\pi(2n)(l + m)!(n + l)!} \\
        N &= \sqrt{\frac{\kappa^3(2l + 1)(l - m)!(n - l - 1)!}{4\pi(2n)(l + m)!(n + l)!}} \\
        \intertext{Let us finally split the expression for $N$ into a product of two roots to get for a reason which will become clear once construct the final wavefunction:}
        N &= \sqrt{\kappa^3\frac{(n - l - 1)!}{2n(n + l)!}}\sqrt{\frac{(2l + 1)(l - m)!}{4\pi(l + m)!}}
    \end{align}

    Now with the normalization constant solved all that remains is multiplying the solution with $N$ to get the final normalized wavefunction:

    \begin{align}
        \psi(\rho, \theta, \varphi) &= NR\Theta\Phi \\
        \psi(\rho, \theta, \varphi) &= \sqrt{\kappa^3\frac{(n - l - 1)!}{2n(n + l)!}}\sqrt{\frac{(2l + 1)(l - m)!}{4\pi(l + m)!}}e^{-\rho/2}\rho^{l}L^{(2l+1)}_{n-l-1}(\rho)P^{m}_{l}( \cos\theta)e^{im\varphi} 
        \intertext{Rearranging the order of the terms we get:}
        \psi(\rho, \theta, \varphi) &= \sqrt{\kappa^3\frac{(n - l - 1)!}{2n(n + l)!}}e^{-\rho/2}\rho^{l}L^{(2l+1)}_{n-l-1}(\rho)\sqrt{\frac{(2l + 1)(l - m)!}{4\pi(l + m)!}}P^{m}_{l}( \cos\theta)e^{im\varphi} 
        \intertext{The second root expression multiplied by $P^{m}_{l}( \cos\theta)e^{im\varphi}$ is, by definition, a normalized spherical harmonic $Y^{m}_{l}(\theta, \varphi)$ which allows us to
        simplify the notation drastically:}
        \psi(\rho, \theta, \varphi) &= \sqrt{\kappa^3\frac{(n - l - 1)!}{2n(n + l)!}}e^{-\rho/2}\rho^{l}L^{(2l+1)}_{n-l-1}(\rho)Y^{m}_{l}(\theta, \varphi) 
    \end{align}

    When we plug in $\kappa = \frac{2}{na_{0}}$ and recognize that the expression for the wavefunction depends on the three integer constants $n$, $l$ and $m$ and index the solution using 
    them we finally get the solution of the Schrödinger equation for the electron of a hydrogen atom:

    \begin{equation}
        \boxed{
            \psi_{nlm}(\rho, \theta, \varphi) = \sqrt{\left( \frac{2}{na_{0}} \right)^3\frac{(n - l - 1)!}{2n(n + l)!}}e^{-\rho/2}\rho^{l}L^{(2l+1)}_{n-l-1}(\rho)Y^{m}_{l}(\theta, \varphi) 
        }
    \end{equation}

    Where $\rho = \frac{2r}{na_{0}}$, $a_{0} = \frac{4\pi\varepsilon_{0}\hbar^2}{\mu e^2}$ is the reduced Bohr radius, $L^{(2l + 1)}_{n - l - 1}(\rho)$ is an associated Legendre polynomial
    of degree $n - l - 1$ and order $2l + 1$ and $Y^{m}_{l}(\theta, \varphi)$ is a spherical harmonic function of degree $l$ and order $m$ which contains as factors $e^{im\varphi}$ and
    $P^{m}_{l}( \cos\theta)$ which in turn is an associated legendre polynomial of degree $l$ and order $m$ parametrized with respect to the cosine of the polar angle $\theta$. 
    



    
    
    



\end{document}
