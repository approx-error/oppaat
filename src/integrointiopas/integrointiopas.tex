\documentclass{article}%
\usepackage[T1]{fontenc}%
\usepackage[utf8]{inputenc}%
\usepackage{lmodern}%
\usepackage{textcomp}% 
\usepackage{lastpage}%
\usepackage{geometry}%
\geometry{margin=3cm}%
\usepackage[finnish]{babel}%
\usepackage{amsmath}%
\usepackage{amssymb}%
\usepackage{amsthm}%
\usepackage{derivative}%
\usepackage{graphicx}%
\graphicspath{{kuvat/}}
\usepackage{cancel}%
\usepackage{comment}%
\usepackage{listingsutf8}%
\usepackage{color}%
\usepackage{esint}
\usepackage{bigints}

% \documentclass[../integrointiopas.tex]{subfiles}
% \graphicspath{{\subfix{../Kuvat/}}}

% \begin{document}
	
% \end{document}

\numberwithin{equation}{section}
\numberwithin{figure}{section}
\numberwithin{table}{section}	

\newtheorem{theorem}{Lause}[section]
\newtheorem{corollary}{Seuraus}[theorem]
\newtheorem{definition}{Määritelmä}[section]
\newtheorem{remark}{Havainto}[section]

% Unit vector, boldstyle
\newcommand{\unitv}[1]{\mathbf{\hat{#1}}}
% Unit vector, bolstyle (greek)
\newcommand{\unitg}[1]{\boldsymbol{\hat{#1}}}
% Vector, boldstyle
\newcommand{\vtr}[1]{\mathbf{#1}}
% Vector, boldstyle (greek)
\newcommand{\vtg}[1]{\boldsymbol{#1}}

% Vector calculus
% Gradient: use nabla
% Laplacian: use nabla^2
\newcommand{\divergenceop}{\nabla\cdot}
\newcommand{\curlop}{\nabla\times}

% Resiudes
\newcommand{\res}[1]{\underset{#1}{\mathrm{Res}}\,}
% Winding number
\newcommand{\wind}{\mathrm{wind}}

% Inverse Hyperbolic
\newcommand{\arsinh}{\mathrm{arsinh}\,}
\newcommand{\arcosh}{\mathrm{arcosh}\,}
\newcommand{\artanh}{\mathrm{artanh}\,}

% Projektin jako useaan tiedostoon:
\usepackage{subfiles}

%
\title{\huge{\textbf{Summa Summarum: \\ Integrointiopas sinulle, joka säikähdit integrointioppaita}}}%
\author{\LARGE{Juuso Kaarela, Helsingin Yliopisto}}%
\date{}
%
\begin{document}%
\normalsize%
\maketitle%

\begin{figure*}[h!]
	\centering
	\includegraphics[width=0.9\linewidth]{TitlePhoto.png}
\end{figure*}

\pagebreak

\tableofcontents

\pagebreak

\section{Esipuhe}

\subfile{osiot/1_esipuhe}

\pagebreak

\section{Oppaassa käytetyt merkinnät}

\subfile{osiot/2_merkinnat}

\pagebreak

\section{Integraalien perusominaisuuksia}

\subfile{osiot/3_perusominaisuuksia}

\pagebreak

\section{Yhden muuttujan funktioiden integrointi}

\subfile{osiot/4_yhden_muuttujan_funktiot}

\pagebreak

\section{Kahden ja kolmen muuttujan funktioiden integrointi}

\subfile{osiot/5_monen_muuttujan_funktiot}

\pagebreak

\section{Numeerinen integrointi}

\subfile{osiot/6_numeerinen_integrointi}

\pagebreak

\section{Integraalien sovelluksia}

\subfile{osiot/7_sovelluksia}

\pagebreak











%
\end{document}
