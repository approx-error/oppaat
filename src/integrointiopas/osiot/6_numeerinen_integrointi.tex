\documentclass[../integrointiopas.tex]{subfiles}
\graphicspath{{\subfix{../kuvat/}}}

\begin{document}
    
    Numeerisella integroinnilla tarkoitetaan sellaisia integroimismenetelmiä, jotka eivät tuota tarkkoja vaan likimääräisiä ratkaisuja.
    Numeerisesta integroinnista on hyötyä erityisesti, kun laskettava integraali on liian työläs ratkaistavaksi analyyttisesti, tai kun
    tuloksen ei tarvitse olla eksakti vaan tietty tarkkuus riittää. Etenkin jos integrointi suoritetaan osana jonkinlaista tietokonesimulaatiota,
    on likimääräinen tulos hyväksyttävä ja jopa toivottava sillä muuten simulaation suoritusaika voisi kasvaa aivan liian pitkäksi. On myös olemassa
    funktioita, joiden integraalifunktiota ei todistetusti voi ilmaista yksinkertaisena alkeisfunktioiden kombinaationa, vaan ratkaisu on jonkinlainen
    ääretön sarja termejä. Tällöin tarkkaa ratkaisua ei voida edes teoriassa ilmaista yhtenä sievänä lausekkeena, vaan konkreettisten arvojen saamiseksi
    on laskettava jokin määrä sarjan termejä, jolloin ratkaisu on aina likimääräinen.

    Tässä osiossa esitellään neljä tapaa suorittaa jollakin välillä hyvin käyttäytyvän mielivaltaisen funktion numeerinen integrointi, sekä esimerkkejä
    numeerisen integroinnin toteutuksesta tieteellisessä laskennassa usein käytetyillä ohjelmointikielillä: Python, Fortran ja C. Kaikki metodit soveltuvat
    ainoastaan määrättyjen integraalien suorittamiseen.

	\subsection{Suorakaidesääntö}

    Suorakaidesääntö perustuu nimensä mukaisesti suorakaiteisiin ja niiden pinta-alan laskemiseen. Koska määrättyjen integraalien laskeminen voidaan tietyssä
    mielessä tulkita funktion ja x-akselin väliin jäävän pinta-alan laskemiseksi (kunhan huomioidaan että x-akselin alapuolinen pinta-ala lasketaan negatiivisena),
    voidaan integraaleja määrittää arvioimalla käyrän ja x-akselin välistä pinta-alaa jollakin välillä. Pinta-alaa voidaan arvioida monella tavall, mutta suorakaidesääntö
    perustuu nimenomaan pinta-alan arvioimiseen käyttäen suorakulmioita. Kuvassa (X) on esitetty välillä $[a,b]$ määritellyn funktion $f(x)$ kuvaaja sekä suorakaiteita, jotka
    arvioivat kuvaajan ja x-akselin välistä pinta-alaa:

    [KUVA TÄHÄN]

    Integrointia aiemmin opiskellut tunnistaa todennäköisesti kuvan (X) mukaisen konstruktion, sillä integroiminen usein opetetaan näyttämällä oppilaalle kuvia
    yhä paremmista ja paremmista suorakulmioapproksimaatioista kunnes rajalla, jossa suorakuolmioiden leveys menee nollaan saadaan todellinen integraali. Tämä myös
    perustelee, miksi suorakaidesääntö antaa tuloksia, jotka ovat likimääräisesti oikeita verrattuna integraalin todelliseen arvoon.

    Siirrytään seuraavaksi ajattelemaan, kuinka kuvan (X) konstruktiota voitaisiin kuvailla matemaattisesti. Haluaisimme laskea kunkin suorakulmion pinta-alan ja summata
    nämä pinta-alat yhteen integraalin määrittämiseksi. Aloitetaan miettimällä tilannetta, jossa väliä arvioidaan yhdellä suorakulmiolla. Tällöin suorakulmion leveys on
    $b-a$ (yleisessä tapauksessa $|b-a|$, sillä $a$ ja $b$ voivat olla negatiivisia). Mikä on sitten suorakulmion korkeus? Voimme valita sen vapaasti, sillä mikään ei
    sano, että jokin kuvaajan pisteistä olisi toista tärkeämpi. Esimerkin vuoksi valitaan suorakulmion korkeudeksi funktion arvo välin alussa, eli $f(a)$. Nyt suorakulmion
    pinta-ala on $(b-a)f(a)$. On selvää, että vakiofunktiota lukuunottamatta oheinen arvio funktion integraalille välillä $[a,b]$ on hyvin epätarkka. Voimme kuitenkin parantaa
    arviota lisäämällä suorakaiteiden määrää. Arvioidaankin nyt pinta-alaa kahdella suorakulmiolla, jotka ovat leveydeltään puolet välistä $[a,b]$ ja joiden korkeus on funktion
    arvo kunkin välin alkupisteessä. Kokonaispinta-alaksi saadaan: $\frac{b-a}{2}f(a) + \frac{b-a}{2}f(a+(b-a)/2)$. Kunkin suorakulmion leveys on puolet välin pituudesta, eli
    $\frac{b-a}{2}$. Ensimmäisen suorakulmion korkeus on funktion arvo välin alussa: $f(a)$ ja toisen korkeus on funktion arvo välin puolivälissä: $f(a+(b-a)/2)$. Tämä on jo
    parempi arvio pinta-alasta mutta pystymme vielä parempaan. Jaetaan nyt väli suorakulmioihin, joiden lukumäärä on $N$. Nyt yhden suorakulmion leveys $d$ on välin pituus $b-a$ jaettuna
    suorakulmioiden lukumäärällä $N$: $d = \frac{b - a}{N}$. Suorakulmion korkeus taas on funktion arvo kunkin välin alkupisteessä. Mikäli ollaan ensimmäisessä välissä, on korkeus $f(a)$,
    mikäli ollaan toisessa välissä on korkeus $f(a + d)$, mikäli ollaan kolmannessa välissä on korkeus $f(a + 2d)$ ja niin edelleen. Yleisesti mikäli ollaan välissä $n < N$, on korkeus $f(a + nd)$.
    Nyt siis kokonaispinta-alaksi saadaan summa: $d\cdot f(a) + d\cdot f(a + d) + d\cdot f(a + 2d) + \dots + d \cdot f(a + Nd)$. Kun tämä ilmaistaan summanotaatiolla, saadaan:

    \begin{equation*}
        A = \sum_{n = 0}^{N}d\cdot f(a + nd)
    \end{equation*}

    Kirjoitetaan vielä $d$:n lauseke auki:

    \begin{equation*}
        A = \sum_{n = 0}^{N}d\cdot f\left( a + n\frac{|b - a|}{N} \right)
    \end{equation*}

	
	\subsection{Puolisuunnikassääntö}
	
	\subsection{Simpsonin sääntö}
	
	Suorakaidesäännön ja puolisuunnikassäännön yhdistelmä
	
	\subsection{Monte Carlo -integrointi}
	
	\subsection{Numeerisen integroinnin toteuttaminen ohjelmoimalla}

    \subsubsection{Suorakaidesääntö Pythonilla}

    Suorakaidesääntö toteutettuna Pythonilla:

    \lstset{style=python-style}
    \lstinputlisting[breaklines]{../koodit/sk\_saanto.py}

    \subsubsection{Puolisuunnikassääntö Fortranilla}

    \subsubsection{Simpsonin sääntö C:llä}

    \subsubsection{Monte Carlo -integrointi }
\end{document}
